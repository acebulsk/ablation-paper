% Options for packages loaded elsewhere
% Options for packages loaded elsewhere
\PassOptionsToPackage{unicode}{hyperref}
\PassOptionsToPackage{hyphens}{url}
\PassOptionsToPackage{dvipsnames,svgnames,x11names}{xcolor}
%
\documentclass[
]{agujournal2019}
\usepackage{xcolor}
\usepackage{amsmath,amssymb}
\setcounter{secnumdepth}{5}
\usepackage{iftex}
\ifPDFTeX
  \usepackage[T1]{fontenc}
  \usepackage[utf8]{inputenc}
  \usepackage{textcomp} % provide euro and other symbols
\else % if luatex or xetex
  \usepackage{unicode-math} % this also loads fontspec
  \defaultfontfeatures{Scale=MatchLowercase}
  \defaultfontfeatures[\rmfamily]{Ligatures=TeX,Scale=1}
\fi
\usepackage{lmodern}
\ifPDFTeX\else
  % xetex/luatex font selection
\fi
% Use upquote if available, for straight quotes in verbatim environments
\IfFileExists{upquote.sty}{\usepackage{upquote}}{}
\IfFileExists{microtype.sty}{% use microtype if available
  \usepackage[]{microtype}
  \UseMicrotypeSet[protrusion]{basicmath} % disable protrusion for tt fonts
}{}
\usepackage{setspace}
\makeatletter
\@ifundefined{KOMAClassName}{% if non-KOMA class
  \IfFileExists{parskip.sty}{%
    \usepackage{parskip}
  }{% else
    \setlength{\parindent}{0pt}
    \setlength{\parskip}{6pt plus 2pt minus 1pt}}
}{% if KOMA class
  \KOMAoptions{parskip=half}}
\makeatother
% Make \paragraph and \subparagraph free-standing
\makeatletter
\ifx\paragraph\undefined\else
  \let\oldparagraph\paragraph
  \renewcommand{\paragraph}{
    \@ifstar
      \xxxParagraphStar
      \xxxParagraphNoStar
  }
  \newcommand{\xxxParagraphStar}[1]{\oldparagraph*{#1}\mbox{}}
  \newcommand{\xxxParagraphNoStar}[1]{\oldparagraph{#1}\mbox{}}
\fi
\ifx\subparagraph\undefined\else
  \let\oldsubparagraph\subparagraph
  \renewcommand{\subparagraph}{
    \@ifstar
      \xxxSubParagraphStar
      \xxxSubParagraphNoStar
  }
  \newcommand{\xxxSubParagraphStar}[1]{\oldsubparagraph*{#1}\mbox{}}
  \newcommand{\xxxSubParagraphNoStar}[1]{\oldsubparagraph{#1}\mbox{}}
\fi
\makeatother


\usepackage{longtable,booktabs,array}
\usepackage{calc} % for calculating minipage widths
% Correct order of tables after \paragraph or \subparagraph
\usepackage{etoolbox}
\makeatletter
\patchcmd\longtable{\par}{\if@noskipsec\mbox{}\fi\par}{}{}
\makeatother
% Allow footnotes in longtable head/foot
\IfFileExists{footnotehyper.sty}{\usepackage{footnotehyper}}{\usepackage{footnote}}
\makesavenoteenv{longtable}
\usepackage{graphicx}
\makeatletter
\newsavebox\pandoc@box
\newcommand*\pandocbounded[1]{% scales image to fit in text height/width
  \sbox\pandoc@box{#1}%
  \Gscale@div\@tempa{\textheight}{\dimexpr\ht\pandoc@box+\dp\pandoc@box\relax}%
  \Gscale@div\@tempb{\linewidth}{\wd\pandoc@box}%
  \ifdim\@tempb\p@<\@tempa\p@\let\@tempa\@tempb\fi% select the smaller of both
  \ifdim\@tempa\p@<\p@\scalebox{\@tempa}{\usebox\pandoc@box}%
  \else\usebox{\pandoc@box}%
  \fi%
}
% Set default figure placement to htbp
\def\fps@figure{htbp}
\makeatother


% definitions for citeproc citations
\NewDocumentCommand\citeproctext{}{}
\NewDocumentCommand\citeproc{mm}{%
  \begingroup\def\citeproctext{#2}\cite{#1}\endgroup}
\makeatletter
 % allow citations to break across lines
 \let\@cite@ofmt\@firstofone
 % avoid brackets around text for \cite:
 \def\@biblabel#1{}
 \def\@cite#1#2{{#1\if@tempswa , #2\fi}}
\makeatother
\newlength{\cslhangindent}
\setlength{\cslhangindent}{1.5em}
\newlength{\csllabelwidth}
\setlength{\csllabelwidth}{3em}
\newenvironment{CSLReferences}[2] % #1 hanging-indent, #2 entry-spacing
 {\begin{list}{}{%
  \setlength{\itemindent}{0pt}
  \setlength{\leftmargin}{0pt}
  \setlength{\parsep}{0pt}
  % turn on hanging indent if param 1 is 1
  \ifodd #1
   \setlength{\leftmargin}{\cslhangindent}
   \setlength{\itemindent}{-1\cslhangindent}
  \fi
  % set entry spacing
  \setlength{\itemsep}{#2\baselineskip}}}
 {\end{list}}
\usepackage{calc}
\newcommand{\CSLBlock}[1]{\hfill\break\parbox[t]{\linewidth}{\strut\ignorespaces#1\strut}}
\newcommand{\CSLLeftMargin}[1]{\parbox[t]{\csllabelwidth}{\strut#1\strut}}
\newcommand{\CSLRightInline}[1]{\parbox[t]{\linewidth - \csllabelwidth}{\strut#1\strut}}
\newcommand{\CSLIndent}[1]{\hspace{\cslhangindent}#1}



\setlength{\emergencystretch}{3em} % prevent overfull lines

\providecommand{\tightlist}{%
  \setlength{\itemsep}{0pt}\setlength{\parskip}{0pt}}



 


\usepackage{url} %this package should fix any errors with URLs in refs.
\usepackage{lineno}
\usepackage[inline]{trackchanges} %for better track changes. finalnew option will compile document with changes incorporated.
\usepackage{soul}
\linenumbers
\makeatletter
\@ifpackageloaded{caption}{}{\usepackage{caption}}
\AtBeginDocument{%
\ifdefined\contentsname
  \renewcommand*\contentsname{Table of contents}
\else
  \newcommand\contentsname{Table of contents}
\fi
\ifdefined\listfigurename
  \renewcommand*\listfigurename{List of Figures}
\else
  \newcommand\listfigurename{List of Figures}
\fi
\ifdefined\listtablename
  \renewcommand*\listtablename{List of Tables}
\else
  \newcommand\listtablename{List of Tables}
\fi
\ifdefined\figurename
  \renewcommand*\figurename{Figure}
\else
  \newcommand\figurename{Figure}
\fi
\ifdefined\tablename
  \renewcommand*\tablename{Table}
\else
  \newcommand\tablename{Table}
\fi
}
\@ifpackageloaded{float}{}{\usepackage{float}}
\floatstyle{ruled}
\@ifundefined{c@chapter}{\newfloat{codelisting}{h}{lop}}{\newfloat{codelisting}{h}{lop}[chapter]}
\floatname{codelisting}{Listing}
\newcommand*\listoflistings{\listof{codelisting}{List of Listings}}
\makeatother
\makeatletter
\makeatother
\makeatletter
\@ifpackageloaded{caption}{}{\usepackage{caption}}
\@ifpackageloaded{subcaption}{}{\usepackage{subcaption}}
\makeatother
\usepackage{bookmark}
\IfFileExists{xurl.sty}{\usepackage{xurl}}{} % add URL line breaks if available
\urlstyle{same}
\hypersetup{
  pdftitle={Processes Governing the Ablation of Intercepted Snow},
  pdfauthor={Alex C. Cebulski; John W. Pomeroy},
  colorlinks=true,
  linkcolor={blue},
  filecolor={Maroon},
  citecolor={Blue},
  urlcolor={Blue},
  pdfcreator={LaTeX via pandoc}}


\journalname{Water Resources Research}

\draftfalse

\begin{document}
\title{Processes Governing the Ablation of Intercepted Snow}

\authors{Alex C. Cebulski\affil{1}, John W. Pomeroy\affil{1}}
\affiliation{1}{Centre for Hydrology, University of Saskatchewan,
Canmore, Canada, }
\correspondingauthor{Alex C. Cebulski}{alex.cebulski@usask.ca}

\begin{keypoints}
\item Unloading was found to be strongly associated with canopy snow
load, wind shear stress, and canopy snowmelt. \item Implementation of a
new model that includes both melt and dry snow unloading processes
improved canopy snow load predictions. \item The fraction of snowfall
sublimated in the canopy differed by up to a factor of three across
existing models. 
\end{keypoints}

\begin{abstract}
Interception and ablation of snow in forest canopies significantly
influence the quantity, timing, and phase of precipitation that reaches
the ground in cold regions forests. Yet current modelling approaches
have uncertain transferability across differing climate and forest
types, often omit key processes, and typically couple interception and
ablation processes in ways that limit both process representation and
evaluation. Here, in-situ observations from a needleleaf forest in the
Canadian Rockies are utilised to evaluate the theories underpinning
existing canopy snow ablation models and develop novel understanding to
support the development of a new canopy snow ablation model. The
observations revealed that canopy snow load, wind shear stress, and
canopy snowmelt were strongly associated with unloading; however, air
temperature and sublimation were not. A new canopy snow ablation model
was developed based on these associations and their impact on the canopy
snow energy and mass balance. This model demonstrated improved
performance in simulating canopy snow load relative to previous
approaches, especially for melt- and wind-dominated ablation events. The
improved performance in representing canopy snow load compared to
existing models results from including energy balance-based melt and dry
snow unloading relationships with snow load and wind shear stress. The
inclusion of both melt and dry snow unloading processes in the new model
also led to more consistent partitioning of snowfall to the atmosphere
versus the ground compared to existing approaches across a wide range of
meteorologies.
\end{abstract}





\setstretch{1.5}
\section{Introduction}\label{introduction}

The seasonal snowpack is an essential component of global water
resources (Immerzeel et al., 2020; Viviroli et al., 2020) and is
increasingly threatened by rapid climate and land cover change
(Aubry-Wake \& Pomeroy, 2023; Fang \& Pomeroy, 2023; López-Moreno et
al., 2014; Szczypta et al., 2015). Vegetation cover can significantly
alter the quantity (Essery et al., 2003; Pomeroy \& Gray, 1995;
Sanmiguel-Vallelado et al., 2020) and timing (Ellis et al., 2010; Safa
et al., 2021) of snow that reaches the ground. Forest canopies cover
more than half of the snow-covered area in the Northern Hemisphere (Kim
et al., 2017), making forest-snow interactions crucial to understand for
informed ecological, land management, and water resource
decision-making. Needleleaf canopies are particularly effective at
intercepting snowfall (Cebulski \& Pomeroy, 2025b; Hedstrom \& Pomeroy,
1998; Pomeroy \& Schmidt, 1993; Storck et al., 2002), where intercepted
snow is subject to increased energy inputs relative to the subcanopy
snowpacks, leading to increased rates of melt, unloading, and/or
sublimation (Parviainen \& Pomeroy, 2000; Pomeroy et al., 1998b; Roesch
et al., 2001; Storck et al., 2002). The partitioning of snow to the
atmosphere via sublimation (Parviainen \& Pomeroy, 2000; Pomeroy et al.,
1998b) or to the ground through unloading and meltwater drip (Lumbrazo
et al., 2022; Roesch et al., 2001; Storck et al., 2002) is highly
sensitive to meteorological conditions and forest structure contributing
to substantial variability in subcanopy snowpacks across regions and
snowfall events. Coastal humid environments typically exhibit small
sublimation losses and a larger influence of unloading and melt (Floyd,
2012; Storck et al., 2002). Here, enhanced canopy energy inputs combined
with high humidity increases both solid snow unloading and melt of snow
intercepted in the canopy (Lumbrazo et al., 2022; Lundquist et al.,
2021; Roesch et al., 2001). Conversely, the colder and drier winters in
continental climates typical of the boreal forests can induce
substantial canopy sublimation losses (e.g., 25---45\% of annual
snowfall in Essery et al., 2003) in addition to unloading (Essery et
al., 2003; Gelfan et al., 2004; Parviainen \& Pomeroy, 2000; Pomeroy et
al., 1998b, 2002). As a result, reliable models of snow accumulation and
streamflow in forested basins rely on a comprehensive understanding of
interception processes (Clark et al., 2015; Essery et al., 2003; Pomeroy
et al., 1998a; Verseghy, 2017; Wheater et al., 2022).

Canopy snow models have demonstrated variable transferability across
different climates and forest types (Essery et al., 2003; Gelfan et al.,
2004; Lumbrazo et al., 2022; Lundquist et al., 2021), and uncertainty in
transferability has been attributed as a key area limiting the
performance in predicting forest snowpacks in hydrological model
intercomparisons (Krinner et al., 2018; Rutter et al., 2009). Recent
studies have emphasized the importance of distinguishing between initial
snow interception and subsequent ablation processes (Cebulski \&
Pomeroy, 2025a; Cebulski \& Pomeroy, 2025b). This separation allows for
individual parameterisations for distinct processes, improving both
process representation and the modular design of contemporary models,
thereby supporting broader applicability across diverse environments and
model structures (Clark et al., 2015; Pomeroy et al., 2022). In
addition, Lundquist et al. (2021) and Cebulski \& Pomeroy (2025b)
demonstrated that canopy snow processes can be more accurately
represented without the concept of a maximum canopy snow load, which is
included in initial accumulation parameterisations (Andreadis et al.,
2009; Hedstrom \& Pomeroy, 1998). Existing ablation routines were
integrated in canopy snow accumulation parameterisations, likely as a
result of some ablation included in measurements of interception
efficiency (Cebulski \& Pomeroy, 2025a). For example, Hedstrom \&
Pomeroy (1998) found that interception efficiency declined as the canopy
is loaded with snow, while Storck et al. (2002) found a constant
interception efficiency of 0.6 and is also typically combined with a
maximum canopy snow load. Consequently, both the Hedstrom \& Pomeroy
(1998) and Storck et al. (2002) parameterisations have significantly
lower interception efficiencies---prior to canopy snow
ablation---compared to the observations by Cebulski \& Pomeroy (2025b).
Staines \& Pomeroy (2023) and Cebulski \& Pomeroy (2025b) show that
interception efficiency is best predicted by canopy density
alone---consistent with some rainfall interception parameterisations
(e.g., Valante et al., 1997; Zhong et al., 2022)---and that snow load
has a small influence on canopy density, challenging the Hedstrom \&
Pomeroy (1998) and Storck et al. (2002) initial snow interception
theories. Together these studies (Cebulski \& Pomeroy, 2025a; Cebulski
\& Pomeroy, 2025b; Lundquist et al., 2021) emphasize the need to revisit
canopy snow ablation parameterisations.

Ablation of intercepted snow due to snow unloading to the ground has
previously been shown to be associated air temperature (Katsushima et
al., 2023; Roesch et al., 2001; Schmidt \& Pomeroy, 1990), ice-bulb
temperature---which more accurately represents the cooling effect of
sublimation compared to air temperature---(Ellis et al., 2010; Floyd,
2012), canopy snowmelt rate (Storck et al., 2002), and wind speed as
shown in Figure~\ref{fig-unl-ex-wind} for (Bartlett \& Verseghy, 2015;
Katsushima et al., 2023; Roesch et al.; 2001, 2001). Each of these
factors were also found to be dependent on canopy snow load and time
(Figure~\ref{fig-unl-ex-wind}). While Hedstrom \& Pomeroy (1998) did not
make direct observations of canopy snow unloading, they proposed a
parameterisation based on canopy snow load and time. In addition to the
empirical observations by the afformentioned studies, physical reasoning
also supports the inclusion of these processes. For example, melt
promotes unloading through loss of structural integrity, particle bond
weakening, and lubrication of intercepted snow. Wind drag promotes
unloading through shear stress applied to intercepted snow, wind erosion
through direct entrainment in the atmosphere of intercepted snow, and
branch movement. Branch elasticity increases with temperature (Schmidt
\& Pomeroy, 1990) which can increase the likelihood of unloading due to
increasing branch angle under a snow load and decreased stiffness to
resist swaying in a turbulent wind.

Melt of snow intercepted in the canopy is typically represented by
either an energy balance approach (Andreadis et al., 2009; Clark et al.,
2015; e.g., Parviainen \& Pomeroy, 2000), as a function of air
temperature (Roesch et al., 2001), or a function of ice-bulb temperature
(Ellis et al., 2010; Floyd, 2012) (Figure~\ref{fig-unl-ex-temp}).
Sublimation is generally represented using a coupled energy and mass
balance approach (e.g., Essery et al., 2003; Pomeroy et al., 1998b;
Verseghy, 2017). The Essery et al. (2003) and Pomeroy et al. (1998b)
approaches differ in that Pomeroy et al. (1998b) does not include
longwave radiation in the canopy snow energy balance. Both the Essery et
al. (2003) and Pomeroy et al. (1998b) parameterisations decrease the
latent heat flux from snow intercepted in the canopy as the canopy fills
up with snow and its specific surface area decreases. However, these
parameterisations are based on estimates of maximum canopy snow load
which may underestimate true maxima (Cebulski \& Pomeroy, 2025b;
Lundquist et al., 2021; Storck et al., 2002) and should be reconsidered
using a larger maximum load or an approach that avoids prescribing a
maximum load. The merits of including more physically based energy
balance methods compared to more empirically based functions for
calculating snowmelt and sublimation have not been directly assessed
using an event-based process investigation.

Quantifying individual canopy snow ablation processes including
unloading, wind redistribution, melt, drip, and sublimation remains
challenging, even with sophisticated lysimetry (Storck et al., 2002) and
eddy-covariance systems (Conway et al., 2018; Harding \& Pomeroy, 1996;
Harvey et al., 2025). Consequently, some canopy snow ablation
parameterisations have been developed using methods, such as above
canopy albedo, which do not distinguish individual processes (Bartlett
\& Verseghy, 2015; Roesch et al., 2001). While these approaches offer
useful indices of model performance, they provide limited insight into
the accuracy of individual process representations. Lysimeter-based
measurements offer more direct process-level observations, but
interpretation of their observations is made uncertain by freeze--thaw
cycles and concurrent processes (Floyd, 2012; MacDonald, 2010; Storck et
al., 2002). A hybrid diagnostic approach that combines individual
process measurements with simulations and employs observations such as
canopy snow load from a weighed tree has yet to be applied but is
explored in this study.

The objective of this study is to better understand and predict the
influences of meteorology and snow load on intercepted snow ablation.
This study specifically looks at canopy snow ablation after snowfall and
initial interception. Cebulski \& Pomeroy (2025b) has addressed
processes governing the initial accumulation of snow in the canopy.

The specific research questions this research aims to address include:

\begin{enumerate}
\def\labelenumi{\arabic{enumi}.}
\item
  How do air temperature, humidity, wind exposure, canopy snow
  sublimation, and snowmelt influence the rate of canopy snow unloading?
\item
  To what extent do current theoretical models of canopy snow ablation
  align with detailed in-situ observations?
\item
  What modifications to existing models, are necessary to accurately
  represent ablation of snow intercepted in the canopy and what is the
  improvement in performance from these modifications?
\end{enumerate}

\begin{figure}[htbp]

\centering{

\pandocbounded{\includegraphics[keepaspectratio]{figs/final/figure1.png}}

}

\caption{\label{fig-unl-ex-wind}The Roesch et al. (2001) model of
unloading rate with increasing wind speed and canopy snow load (left,
R01) and the Hedstrom \& Pomeroy (1998) model of unloading rate with
increasing snow load (right, E10). Both examples have a constant air
temperature of -10°C to disable the influence of warming on unloading
and drip.}

\end{figure}%

\begin{figure}[htbp]

\centering{

\pandocbounded{\includegraphics[keepaspectratio]{figs/final/figure2.png}}

}

\caption{\label{fig-unl-ex-temp}The Ellis et al. (2010) and Floyd (2012)
(E10) model of unloading and drip rate (left) and the Roesch et al.
(2001) (R01) model of unloading rate (right) with increasing air
temperature. Wind speed for the R01 parameterisation was set to zero.}

\end{figure}%

\section{Methods}\label{methods}

\subsection{Study Site}\label{study-site}

The observations presented in this study were collected at Fortress
Mountain Research Basin (FMRB), Alberta, Canada, -115° W, 51° N, a
continental headwater basin in the Canadian Rockies. The site is located
2100 m above sea level on a ridge covered by mature fir and spruce
forest. Air temperature, humidity, and wind speed were measured at a
height of 4.3 m at Forest Tower (FT) Station
(Figure~\ref{fig-site-map}). Shear stress was calculated using the
EddyPro software (LI-COR Biosciences) based on high-frequency wind
measurements from a CSAT3 three-dimensional sonic anemometer (Campbell
Scientific) installed at 3.0 m at FT station. The CSAT3 was occasionally
covered in snow during the analysis period, and thus to provide a
complete record of shear stress, a linear relationship was established
between shear stress derived from the CSAT3 and the square of wind speed
measured at 4.3 m at the FT station (\emph{R\textsuperscript{2}} = 0.71,
\emph{p} \textless{} 0.05). This relationship was then used to gap-fill
shear stress during periods when the CSAT3 was snow-covered. The
precipitation rate was measured by an Alter-shielded OTT Pluvio weighing
precipitation gauge installed 2.6 m above ground at the adjacent
Powerline (PWL) Station (Figure~\ref{fig-site-map}). Incoming and
outgoing solar radiation was measured by a Kipp \& Zonen CNR4
4-Component Net Radiometer installed 3.27 m above the ground at Fortress
Ridge Station (FRS) 2.0 km to the northwest of FT station (-115.2° W,
50.8° N). This windy exposed site was selected to reduce snow
accumulation on the radiometers, in addition to the CNR4's built-in
heating element. Three subcanopy lysimeters, consisting of plastic
horse-watering troughs with an opening of 0.9 m\textsuperscript{2} and
depth of 20 cm suspended from a load cell, were installed to measure
subcanopy snow accumulation. A weighed tree lysimeter (subalpine fir)
suspended from a load cell (Artech S-Type 20210-100) measured the weight
of canopy snow load (kg) and was scaled to an areal estimate of snow
load (mm) using snow surveys as in Pomeroy \& Schmidt (1993). To isolate
the subcanopy lysimeter measurements of canopy snow unloading from
throughfall, 15-min intervals were selected that had no atmospheric
precipitation based on the precipitation gauge at PWL. The weighed tree
and timelapse imagery also were used to confirm there was no atmospheric
precipitation and to identify periods where snow was intercepted in the
canopy and ablation was possible. Four tipping bucket rain gauges---3
Texas Electronics TR-525M and 1 Hyquest Solutions TB4MM---were installed
along a 15 m transect adjacent to the dense canopy subcanopy lysimeter
to quantify sub-canopy drip from melting intercepted snow. Additional
details on this study site and the meteorological and lysimeter
measurements have been described in Cebulski \& Pomeroy (2025b).

\begin{figure}[htbp]

\centering{

\pandocbounded{\includegraphics[keepaspectratio]{figs/final/figure3.png}}

}

\caption{\label{fig-site-map}Map showing the location of flux towers and
lysimeter instruments. The inset map on the upper left shows the
regional location of Fortress Mountain Research basin. Flux towers are
denoted by a circle, lysimeters by a square, and tipping bucket rain
gauges (TB) by diamonds.}

\end{figure}%

\subsection{The Cold Regions Hydrological Model
Platform}\label{the-cold-regions-hydrological-model-platform}

The Cold Regions Hydrological Model Platform (CRHM) was used to
implement calculations of the canopy snow energy and mass budget. A full
description of the CRHM platform is described in Pomeroy et al. (2022)
and the source code is available at
https://github.com/srlabUsask/crhmcode. The climate forcing data used to
run CRHM was from station-based fifteen-minute interval measurements of
air temperature, relative humidity, wind speed, precipitation, and
incoming solar radiation from the FT, PWL, and FRS stations. CRHM
incorporates a flexible modular design allowing the user to select
various modules (parameterisations) that represent hydrological
processes. The phase of atmospheric precipitation was determined from
the energy balance of falling hydrometeors (Harder \& Pomeroy, 2013). A
new CRHM module was created here to simulate the coupled mass and energy
balance of snow intercepted in the canopy. The energy balance is
described in detail in Section~\ref{sec-ebal} and the updates to the
mass balance through revisions to the canopy snow unloading empirical
functions are presented in Section~\ref{sec-dry-unld} and
Section~\ref{sec-melt-unld}.

\subsection{Canopy Snow Mass Balance}\label{sec-mbal}

The following mass balance as described in Cebulski \& Pomeroy (2025a)
was implemented in CRHM to track the state of canopy snow load (\(L\),
mm) over time:

\begin{equation}\phantomsection\label{eq-canopy-mass-bal}{
\frac{dL}{dt} = 
[q_{sf} - q_{tf} + q_{ros}] - [q_{unld}^{melt} + q_{unld}^{dry}] - q_{drip} - q_{wind}^{veg} - q_{sub}^{veg}
}\end{equation}

where \(\frac{dL}{dt}\) is the change in canopy snow load over time, (mm
s\textsuperscript{-1}), \(q_{sf}\) is the snowfall rate (mm
s\textsuperscript{-1}), \(q_{tf}\) is the throughfall rate (mm
s\textsuperscript{-1}), \(q_{ros}\) is the rate of rainfall falling on
snow intercepted in the canopy (mm s\textsuperscript{-1}),
\(q_{unld}^{melt}\) is the unloading rate due to melt (mm
s\textsuperscript{-1}), \(q_{unld}^{dry}\) is the dry snow unloading
rate due to shear stress on snow, wind erosion, branch movement,
structural degradation, bond weakening and increased elasticity of
branches and other non-melt related processes (mm
s\textsuperscript{-1}), \(q_{drip}\) is the canopy snow drip rate (mm
s\textsuperscript{-1}) resulting from canopy snowmelt (\(q_{melt}\), mm
s\textsuperscript{-1}), \(q_{wind}^{veg}\) is the wind transport rate in
or out of the control volume (mm s\textsuperscript{-1}), and
\(q_{sub}^{veg}\) is the intercepted snow sublimation rate (mm
s\textsuperscript{-1}) which includes any evaporation of liquid water in
the canopy. Figure 1 in Cebulski \& Pomeroy (2025a) presents a visual
representation of this mass balance.

\subsection{Mass Balance
Parameterisations}\label{mass-balance-parameterisations}

In addition to the updated canopy snow model presented in this study,
hereafter referred to as CP25 (described in Section~\ref{sec-dry-unld}
and Section~\ref{sec-melt-unld}), three other canopy snow models were
implemented in CRHM following previous studies by Ellis et al. (2010)
and Floyd (2012) (E10), Roesch et al. (2001) (R01), and Andreadis et al.
(2009) who built on observations by Storck et al. (2002) (SA09). The E10
model includes canopy snow sublimation as described in Pomeroy et al.
(1998b), dry snow unloading as a function of canopy snow load from
Hedstrom \& Pomeroy (1998) with modifications described in Ellis et al.
(2010) and Floyd (2012) to handle canopy snow melt and drip processes
using an ice-bulb temperature threshold. The R01 model represents dry
snow unloading, melt, and drip using linear functions of wind speed and
air temperature, while sublimation is simulated using the
parameterisation from Pomeroy et al. (1998b). The SA09 model unloads
snow as a ratio of the canopy snowmelt rate, following observations by
Storck et al. (2002) and does not include dry snow unloading. The
snowmelt rate from Equation~\ref{eq-eb} was used to calculate the
unloading rate for the Andreadis et al. (2009) unloading and thus
differs from the energy balance routine described in their study. Canopy
snow sublimation in SA09 was represented based on the Essery et al.
(2003) parameterisation (Equation~\ref{eq-ql}) as implemented in CP25.

The retention of canopy snow meltwater differs between the four canopy
snow models implemented in CRHM. For E10, liquid meltwater is not
retained in the canopy and immediately drips before it can evaporate.
The R01 model also does not represent evaporation of liquid meltwater as
canopy snow is not explicitly separated between solid snow unloading and
melt/drip. The canopy liquid water storage capacity (\(L_{max}^{liq}\),
mm) from SA09 and CP25 was calculated as:

\begin{equation}\phantomsection\label{eq-liq-hold-cap}{
L_{max}^{liq} = Lh + nLAI
}\end{equation}

where \(h\) is the liquid meltwater holding capacity that canopy can
retain against gravity and \(n\) is a storage constant of the vegetation
elements. Andreadis et al. (2009) estimated \(h\) as 0.035 (-) and \(n\)
as \(e^{-4}\). In this study, \(h\) was set to 0.01 (-), and \(n\) was
defined as \(C_c\cdot 0.2\). A one-sided effective plant (leaf) area
index (LAI) was used, including the clumping factor to account for
needleleaf canopy structures, following the approach of Ellis et al.
(2010) for rainfall interception.

\subsection{Canopy Snow Energy Balance}\label{sec-ebal}

The CP25 and SA09 canopy snow models implemented the following energy
balance approach to calculate the energy available for melting snow
intercepted in the canopy (\(Q_{melt}\), W m\textsuperscript{-2}) and to
track the canopy snow temperature over time
(\(\frac{\Delta T_{vs}}{\Delta t}\), K s\textsuperscript{-1}). The
energy balance is expressed as:

\begin{equation}\phantomsection\label{eq-eb}{
Q_{melt} = 
Q_{sw} +
Q_{lw} +
Q_{p} + Q_{h} + Q_{l} - [C_p^{ice} L \frac{\Delta T_{vs}}{\Delta t}]
}\end{equation}

where \(Q_{sw}\) and \(Q_{lw}\) (W m\textsuperscript{-2}) are the net
shortwave and longwave radiation heat fluxes to the canopy snow, \(Q_p\)
(W m\textsuperscript{-2}), is the advective energy rate, and \(Q_{l}\)
and \(Q_{h}\) (W m\textsuperscript{-2}), are the turbulent fluxes of
latent heat and sensible heat respectively (positive towards canopy
snow), and \(C_p^{ice}\) (J kg\textsuperscript{-1}
K\textsuperscript{-1}) is the specific heat capacity of ice. Figure 2 in
Cebulski \& Pomeroy (2025a) shows a visual representation of this energy
balance.

\subsection{Energy Balance
Parameterisations}\label{energy-balance-parameterisations}

The E10 and R01 parameterisations relied on physically guided empirical
relationships to simulate sublimation and melt and are described in full
detail in their respective articles and are also summarised in Cebulski
\& Pomeroy (2025a). The following section describes the energy balance
parameterisations implemented in the CP25 and SA09 models.

\(Q_{sw}\) was determined as:

\begin{equation}\phantomsection\label{eq-sw}{
Q_{sw} = Q_{sw}^{in} \cdot (1 - \alpha_s) \cdot \tau_{50}^{veg}
}\end{equation}

where \(Q_{sw}^{in}\) is the downwelling shortwave radiation (W
m\textsuperscript{-2}), \(\alpha_s\) is the albedo of snow intercepted
in the canopy (-), \(\tau_{50}^{veg}\) (-) is the canopy transmittance
to \(Q_{sw}^{in}\). Equation 10 from Pomeroy et al. (2009) was used to
determine \(\tau_{50}^{veg}\), using half of the leaf area index (LAI)
based on studies Weiskittel et al. (2009) and Kesselring et al. (2024)
that approximately 50\% of the total leaf area is concentrated in the
upper half of coniferous canopies. \(Q_{sw}\) provides an approximation
of the net shortwave radiation to all snow intercepted in the canopy and
is a simplification from using a partial differential equation to
determine the radiation incident to individual height layers within the
canopy. Upwelling shortwave radiation reflected off the surface snowpack
is considered negligible contribution to the snow intercepted in the
canopy as it is primarily blocked by vegetation elements underlying the
canopy snow (Pomeroy et al., 2009).

\(Q_{lw}\) was approximated as:

\begin{equation}\phantomsection\label{eq-lw}{
Q_{lw} = \downarrow Q_{lw}^{atm} + \uparrow Q_{lw}^{veg} - Q_{lw}^{vs}
}\end{equation}

where \(Q_{lw}^{atm}\) is the downwelling longwave radiation from the
atmosphere (W m\textsuperscript{-2}), \(Q_{lw}^{veg}\) is the longwave
radiation upwelling from vegetation elements underlying snow intercepted
in the canopy (W m\textsuperscript{-2}), and \(Q_{lw}^{vs}\) (W
m\textsuperscript{-2}) is the outgoing longwave radiation from the top
and bottom of the canopy snow layer calculated as:

\begin{equation}\phantomsection\label{eq-lw-vs}{
Q_{lw}^{vs} = 2 \epsilon_{s} \sigma T_{vs}^4
}\end{equation}

where \(\epsilon_s\) is the emissivity (-) of snow taken as 0.99 and
\(\sigma\) is the Stefan--Boltzmann (5.67e\textsuperscript{-10} W
m\textsuperscript{-1} K\textsuperscript{-4}). \(Q_{lw}^{atm}\) was
approximated in this study as in Sicart et al. (2006) to represent the
influence of atmospheric moisture and clouds on emissivity.
\(Q_{lw}^{veg}\) was calculated with the assumption that canopy elements
are in equilibrium with the air temperature plus any increase in
vegetation temperature from the extinction of \(Q_{sw}^{in}\) in the
canopy (Pomeroy et al., 2009, Eq. 4).

\(Q_p\) was calculated as:

\begin{equation}\phantomsection\label{eq-qp}{
Q_p = [C_p^{liq} m_r(T_r - T_{vs}) + C_p^{ice} m_s(T_s - T_{vs})] / \Delta t
}\end{equation}

where \(C_p^{liq}\) is the specific heat capacity of liquid water (J
kg\textsuperscript{-1} K\textsuperscript{-1}), \(m_r\) is the specific
mass of liquid water in precipitation (mm), \(T_r\) is the rainfall
temperature (K), \(m_s\) is the specific mass of snow in precipitation
(mm), and \(T_s\) is the snowfall temperature (K).

\(Q_h\) was calculated as:

\begin{equation}\phantomsection\label{eq-qh}{
Q_h = \frac{\rho_a}{r_a} C_p^{air} (T_a - T_{vs})
}\end{equation}

where \(\rho_a\) is the air density (kg m\textsuperscript{-3}),
\(C_p^{air}\) is the specific heat capacity of air (J
kg\textsuperscript{-1} K\textsuperscript{-1}), \(T_a\) is the air
temperature, and \(r_a\) is the aerodynamic resistance (s
m\textsuperscript{-1}) which was approximated from Equation 4 in Allan
et al. (1998) as:

\begin{equation}\phantomsection\label{eq-ra}{
r_a = \frac{\text{log}(\frac{z_T - d_0}{z_0})\text{log}(\frac{z_u - d_0}{z_0})}{\kappa^2 u_z}
}\end{equation}

where \(z_T\) is the height of temperature measurement (m), \(d_0\) is
the displacement height (m) which was approximated as
2/3\textsuperscript{rd} the mean canopy height, \(z_0\) is the roughness
length (m) which was approximated as 1/10\textsuperscript{th} of the
mean canopy height, \(z_u\) is the wind speed measurement height (m),
\(\kappa\) is von Kármán's constant, 0.41 (-), and \(u_z\) is the wind
speed measurement at \(z_u\) (m s\textsuperscript{-1}).

\(Q_l\) was calculated as:

\begin{equation}\phantomsection\label{eq-ql}{
Q_l = \frac{\rho_a}{r_i+r_a} (q_a(T_a) - q_{vs}(T_{vs}))
}\end{equation}

where \(r_i\) is a resistance for transport of moisture from intercepted
snow to the canopy air space (Eq. 28 in Essery et al., 2003),
\(q_a(T_a)\) and \(q_{vs}(T_{vs})\) are the specific humidity (-) at the
air temperature and canopy snow temperature, respectively. \(r_i\) was
calculated following the concept of how full the canopy is with snow as
introduced by Pomeroy \& Schmidt (1993) with modifications to
incorporate a larger maximum canopy snow load capacity of 50 mm based on
observations by Storck et al. (2002), Floyd (2012), and Cebulski \&
Pomeroy (2025b).

The above sensible and latent heat flux equations assume neutral
atmospheric stability conditions, which is supported by the uncertainty
of stability correction in forest canopies (Conway et al., 2018) and
mountain environments in winter (Helgason \& Pomeroy, 2012a). Solving
Equation~\ref{eq-eb} requires an iterative solution to determine
\(\Delta T_{vs}\) and the remaining terms which are also a function of
\(T_{vs}\).

\subsection{Influence of Predictive Variables on
Unloading}\label{influence-of-predictive-variables-on-unloading}

The effects of air temperature, wind speed, snow load, melt, and
sublimation on the unloading process were assessed using a multivariate
ordinary least squares regression. The following hypotheses were tested:

\begin{enumerate}
\def\labelenumi{\alph{enumi}.}
\tightlist
\item
  Melt promotes unloading through loss of structural integrity, particle
  bond weakening, and lubrication of intercepted snow.
\item
  Sublimation promotes unloading via structural degradation and bond
  weakening of intercepted snow.
\item
  Wind drag promotes unloading through shear stress applied to
  intercepted snow, wind erosion through direct entrainment in the
  atmosphere of intercepted snow, and branch movement.
\item
  Increasing air temperature promotes unloading by increasing the
  elasticity of branches and its association with melt and/or
  sublimation.
\end{enumerate}

Since these processes occur simultaneously and could not be isolated
experimentally, different combinations of the independent variables were
included in the regression to identify which sets of processes
significantly influenced unloading. The subcanopy lysimeter unloading
measurements had a high relative instrument error due to the relatively
small accumulation of unloaded snow over the 15-min intervals. To
improve instrument accuracy, whilst maintaining consistency of the
unloading measurements with the independent variables, the 15-min
interval measurements of unloading were aggregated over differing
predictive variable bins. Independent variable bins that had less than
0.1 mm of accumulated snow were removed, resulting in a mean instrument
error of +/- 2\% for the remaining bins. Air temperature and wind speed
were measured at the FT station, canopy snow load from the weighed tree
lysimeter (scaled to the canopy of each respective subcanopy lysimeter),
and canopy snowmelt and sublimation simulated using CRHM with the CP25
model as described in the previous section. The individual processes
found to be significant predictors of unloading in the multivariate
regression (i.e., shear stress and canopy snow melt) were isolated to
parameterise a model of their effects that was implemented in the CP25
model.

\subsubsection{Dry Snow Unloading}\label{dry-snow-unloading}

Wind, shear stress, and canopy snow load were assessed as predictors of
dry snow unloading during intervals without canopy snowmelt. These
periods were defined using simulated canopy snowmelt in CRHM as well as
visual analysis of time-lapse imagery for canopy snow drip and/or icicle
formation. The relationships between wind speed, shear stress, canopy
snow load, and unloading were analyzed using linear and non-linear least
squares regressions, linearly with shear stress and exponentially with
wind speed. The linear relationship between shear stress, canopy load,
and unloading did not include an intercept term and was thus the
coefficient of determination (\emph{R\textsuperscript{2}}) was adjusted
following (Kozak \& Kozak, 1995).

\subsubsection{Canopy Snowmelt Induced
Unloading}\label{canopy-snowmelt-induced-unloading}

A mass balance approach was incorporated to determine the unloading rate
resulting from canopy snowmelt (\(q_{unld}^{melt}\), mm
s\textsuperscript{-1}). The effect of the canopy snowmelt rate
(\(q_{melt}\)) on unloading was then assessed by fitting a linear model
using an ordinary least squares regression. Since direct measurements of
canopy snow unloading are challenging to obtain independently from
canopy snowmelt drainage (Storck et al., 2002), the mass balance
introduced in Equation~\ref{eq-canopy-mass-bal} was incorporated to
determine \(q_{unld}^{melt}\) as a residual. During intervals without
\([q_{sf} - q_{tf} + q_{ros}]\) or \(q_{wind}^{veg}\)
Equation~\ref{eq-canopy-mass-bal} was simplified and rearranged to:

\begin{equation}\phantomsection\label{eq-unld-melt-mass-bal}{
q_{unld}^{melt} = -\frac{dL}{dt} - q_{drip} - q_{unld}^{dry} - q_{sub}^{veg}
}\end{equation}

While some components of the canopy snow mass balance can be measured
directly, such as \(\frac{\Delta L}{\Delta t}\) with the weighed tree,
sublimation of canopy snow is more difficult to quantify directly
especially in forested mountain environments (Conway et al., 2018;
Harding \& Pomeroy, 1996; Helgason \& Pomeroy, 2012b; Parviainen \&
Pomeroy, 2000) and thus \(q_{sub}^{veg}\) was simulated in this study in
CRHM using Equation~\ref{eq-ql} as described in Essery et al. (2003).
\(q_{drip}\) was measured where possible using the rain gauges, however
problems with freezing of liquid water in the tipping bucket mechanisms
limited the ability to measure \(q_{drip}\) reliably. Thus, \(q_{drip}\)
was also estimated using simulations of canopy snowmelt (\(q_{melt}\))
in CRHM as in Equation~\ref{eq-eb}, with storage limited by the canopy
liquid water holding capacity and drainage of excess water. Canopy snow
ablation periods that were dominated by melt were selected for
calculating \(q_{unld}^{melt}\) where the contribution of
\(q_{unld}^{dry}\) and \(q_{sub}^{veg}\) to canopy snow ablation was
less than 5\%.

\section{Results}\label{results}

\subsection{Unloading Relationships}\label{sec-unld-rel}

Amongst the models evaluated, a multivariate linear regression
incorporating canopy snow load, canopy snowmelt, and wind speed provided
the highest explanatory power for predicting canopy snow unloading
measured by subcanopy lysimeters (\emph{R\textsuperscript{2}} = 0.79, p
\textless{} 0.05; Table~\ref{tbl-q-unld-bins}). Shear stress was found
to explain less variability compared to wind speed, when both were
combined with canopy snow load and snowmelt (\emph{R\textsuperscript{2}}
= 0.71, \emph{p} \textless{} 0.05). Air temperature and canopy snow
sublimation were not significant predictors in any model (\emph{p}
\textgreater{} 0.05; Table~\ref{tbl-q-unld-bins}). A model including
only canopy load, air temperature, and wind speed produced an
\emph{R\textsuperscript{2}} of 0.11 however, only canopy load and wind
speed were statistically significant (\emph{p} \textless{} 0.05). As
shown in Figure~\ref{fig-q-unld-all-bins}, unloading rates varied more
with snowmelt and sublimation (0--2 mm hr\textsuperscript{-1}) than with
air temperature and wind speed (0--0.5 mm hr\textsuperscript{-1}).

The mean unloading rate was observed to increase with increasing canopy
load, air temperature, ice-bulb temperature depression, shear stress,
and wind speed (Figure~\ref{fig-q-unld-all-bins}). An increase in
unloading was found with sublimation rates between 0---0.3 mm
hr\textsuperscript{-1} (Figure~\ref{fig-q-unld-all-bins}). For
sublimation rates higher than 0.3 mm hr\textsuperscript{-1}, unloading
declined with sublimation. The decline in unloading with wind speed
\textgreater3 m s\textsuperscript{-1} might have been contributed to by
wind transport and entrainment into the atmosphere, and/or increased
sublimation rates at higher wind speeds.

\begin{longtable}[]{@{}
  >{\raggedright\arraybackslash}p{(\linewidth - 18\tabcolsep) * \real{0.1000}}
  >{\raggedright\arraybackslash}p{(\linewidth - 18\tabcolsep) * \real{0.1100}}
  >{\raggedright\arraybackslash}p{(\linewidth - 18\tabcolsep) * \real{0.1100}}
  >{\raggedright\arraybackslash}p{(\linewidth - 18\tabcolsep) * \real{0.1100}}
  >{\raggedright\arraybackslash}p{(\linewidth - 18\tabcolsep) * \real{0.1100}}
  >{\raggedright\arraybackslash}p{(\linewidth - 18\tabcolsep) * \real{0.1100}}
  >{\raggedright\arraybackslash}p{(\linewidth - 18\tabcolsep) * \real{0.1100}}
  >{\raggedright\arraybackslash}p{(\linewidth - 18\tabcolsep) * \real{0.1100}}
  >{\raggedleft\arraybackslash}p{(\linewidth - 18\tabcolsep) * \real{0.0700}}
  >{\raggedleft\arraybackslash}p{(\linewidth - 18\tabcolsep) * \real{0.0800}}@{}}

\caption{\label{tbl-q-unld-bins}Summary of multivariate linear
regression results evaluating all combinations of predictor variables
for canopy snow unloading including: canopy load (\(L\)), wind speed
(\(u\)), canopy snowmelt rate (\(q_{melt}\)), canopy snow sublimation
rate (\(q_{subl}\)), and air temperature (\(T_a\)). Columns \(L\) to
\(T_a\) show the coefficient estimate for each respective term, and the
significance of each term is shown in brackets. Significance codes: * =
p \textless{} 0.05; ns = not significant (p \textgreater{} 0.05). The
models are ranked by their corresponding AIC value.}

\tabularnewline

\toprule\noalign{}
\begin{minipage}[b]{\linewidth}\raggedright
Model Name
\end{minipage} & \begin{minipage}[b]{\linewidth}\raggedright
Intercept
\end{minipage} & \begin{minipage}[b]{\linewidth}\raggedright
\(L\)
\end{minipage} & \begin{minipage}[b]{\linewidth}\raggedright
\(u\)
\end{minipage} & \begin{minipage}[b]{\linewidth}\raggedright
\(q_{melt}\)
\end{minipage} & \begin{minipage}[b]{\linewidth}\raggedright
\(q_{subl}\)
\end{minipage} & \begin{minipage}[b]{\linewidth}\raggedright
\(\tau\)
\end{minipage} & \begin{minipage}[b]{\linewidth}\raggedright
\(T_a\)
\end{minipage} & \begin{minipage}[b]{\linewidth}\raggedleft
\(R^2\)
\end{minipage} & \begin{minipage}[b]{\linewidth}\raggedleft
AIC
\end{minipage} \\
\midrule\noalign{}
\endhead
\bottomrule\noalign{}
\endlastfoot
M1 & -0.11 (ns) & 0.02 (*) & 0.08 (*) & 0.40 (*) & --- & --- & --- &
0.79 & -12.8 \\
M4 & -0.08 (ns) & 0.04 (*) & --- & 0.39 (*) & --- & 0.75 (*) & --- &
0.71 & 5.5 \\
M7 & 0.13 (ns) & 0.02 (*) & --- & 0.32 (*) & -0.22 (ns) & --- & --- &
0.54 & 10.0 \\
M10 & -0.06 (ns) & 0.02 (*) & 0.08 (*) & 0.38 (*) & --- & --- & 0.00
(ns) & 0.52 & -4.4 \\
M24 & -0.00 (ns) & 0.02 (*) & 0.05 (*) & 0.36 (*) & 0.13 (ns) & --- &
--- & 0.37 & -2.0 \\
M40 & 0.07 (ns) & 0.01 (*) & 0.06 (*) & --- & --- & --- & 0.01 (ns) &
0.11 & 2.4 \\
M63 & 0.22 (*) & 0.00 (ns) & -0.01 (ns) & --- & 0.07 (ns) & --- & --- &
-0.02 & 39.8 \\

\end{longtable}

\begin{figure}[htbp]

\centering{

\includegraphics[width=1\linewidth,height=\textheight,keepaspectratio]{figs/final/figure4.png}

}

\caption{\label{fig-q-unld-all-bins}Scatter plots showing the mean
unloading rate (mm hr\textsuperscript{-1}) for differing bins of air
temperature (°C), ice-bulb temperature depression (°C), shear stress (N
m\textsuperscript{-2}), canopy snowmelt (mm hr\textsuperscript{-1}),
canopy snow sublimation (mm hr\textsuperscript{-1}), and wind speed (m
s\textsuperscript{-1}). Note: unloading was measured by the subcanopy
lysimeters, air temperature and wind speed were measured at FT station,
canopy snowmelt and sublimation were calculated using CRHM.}

\end{figure}%

\subsubsection{The Influence of Wind on Dry Snow
Unloading}\label{sec-dry-unld}

Canopy snow unloading measured from the subcanopy lysimeters---filtered
to include intervals without canopy snowmelt---were positively
associated in a linear relationship with shear stress and an exponential
relationship with wind speed (Figure~\ref{fig-q-unld-wind}). The
following equations were fitted to these relationships and tested.

The dry unloading rate, \(q_{unld}^{dry}\), was represented as a linear
function of shear stress:

\begin{equation}\phantomsection\label{eq-q-unld-tau}{
q_{unld}^{dry} = L \cdot \tau_{mid} \cdot a
}\end{equation}

where \(\tau_{mid}\) is the shear stress at mid canopy height and \(a\)
is a fitting constant.

An exponential function of wind speed was defined as:

\begin{equation}\phantomsection\label{eq-q-unld-wind}{
q_{unld}^{dry} = L \cdot u_{mid} \cdot a \cdot e^{b\cdot u_{mid}}
}\end{equation}

where \(u_{mid}\) is the wind speed at mid canopy height, and \(a\) and
\(b\) are fitting constants.

The shear stress relationship (Equation~\ref{eq-q-unld-tau}) accounted
for slightly more variability in unloading (\emph{p} \textless{} 0.05,
\emph{R\textsuperscript{2}} = 0.61) compared to wind speed (\emph{p}
\textless{} 0.05, \emph{R\textsuperscript{2}} = 0.54)
(Table~\ref{tbl-q-unld-wind}). The mean bias of the shear stress model
of 0.037 mm hr\textsuperscript{-1} was also lower compared to the wind
speed model of 0.048 mm hr\textsuperscript{-1}, additional model error
statistics and fitting coefficients are provided in
Table~\ref{tbl-q-unld-wind}. Both models exhibited considerable scatter,
with notable uncertainty resulting from instrument error and processes
other than wind contributing to unloading
(Figure~\ref{fig-q-unld-wind}). The wind-induced unloading rate was
observed to be higher for greater canopy snow loads
(Figure~\ref{fig-q-unld-wind}). The \emph{R\textsuperscript{2}} of both
the shear stress and wind speed relationships was much higher than the
variance explained by wind speed when including intervals with melting
snow (Table~\ref{tbl-q-unld-bins}). The higher
\emph{R\textsuperscript{2}} of the shear stress model compared to wind
speed, coupled with the better physical representation of kinetic
energy, suggests that shear stress be selected as the independent
variable to predict dry snow unloading in the model evaluation.

\begin{figure}[htbp]

\centering{

\pandocbounded{\includegraphics[keepaspectratio]{figs/final/figure5.png}}

}

\caption{\label{fig-q-unld-wind}Canopy snow unloading rate measured by
the subcanopy lysimeters versus shear stress (left) and wind speed
(right) during periods without canopy snowmelt. The dots represent mean
unloading rates within bins of shear stress and wind speed for three
canopy snow load levels; error bars indicate +/- 1 standard deviation.
The fitted lines show predictions from Equation~\ref{eq-q-unld-tau}
(left) and Equation~\ref{eq-q-unld-wind} (right).}

\end{figure}%

\begin{longtable}[]{@{}
  >{\raggedright\arraybackslash}p{(\linewidth - 4\tabcolsep) * \real{0.4458}}
  >{\raggedright\arraybackslash}p{(\linewidth - 4\tabcolsep) * \real{0.2771}}
  >{\raggedright\arraybackslash}p{(\linewidth - 4\tabcolsep) * \real{0.2771}}@{}}

\caption{\label{tbl-q-unld-wind}Summary of regression error statistics
and coefficients for the relationship between canopy snow unloading with
wind speed (Equation~\ref{eq-q-unld-tau}) and shear stress
(Equation~\ref{eq-q-unld-wind}), as shown in
Figure~\ref{fig-q-unld-wind}. Coefficients are shown for hourly
unloading.}

\tabularnewline

\toprule\noalign{}
\begin{minipage}[b]{\linewidth}\raggedright
Metric
\end{minipage} & \begin{minipage}[b]{\linewidth}\raggedright
Wind
\end{minipage} & \begin{minipage}[b]{\linewidth}\raggedright
Shear Stress
\end{minipage} \\
\midrule\noalign{}
\endhead
\bottomrule\noalign{}
\endlastfoot
Mean Bias (mm/hr) & 0.048 & 0.037 \\
Mean Absolute Error (mm/hr) & 0.087 & 0.115 \\
Root Mean Square Error (mm/hr) & 0.11 & 0.15 \\
Coefficient of Determination (\(R^2\)) & 0.54 & 0.61 \\
Coefficient a & \(4.62 \times 10^{-03}\) & \(3.31 \times 10^{-01}\) \\
Significance of a & p \textless{} 0.05 & p \textless{} 0.05 \\
Coefficient b & \(3.93 \times 10^{-01}\) & NA \\
Significance of b & p \textless{} 0.05 & NA \\

\end{longtable}

\subsubsection{The Influence of Melt on Unloading}\label{sec-melt-unld}

Five warm \& humid events were selected, in which the median air
temperature was above 0°C and relative humidity was above 65\%,
resulting in less than 5\% contribution of dry snow unloading and
sublimation to ablation as determined by the CP25 model. For these
events, unloading was calculated using
Equation~\ref{eq-unld-melt-mass-bal} and canopy snowmelt rates were
calculated using CRHM with Equation~\ref{eq-eb}. These unloading
estimates were found to be positively correlated with canopy snowmelt
(Figure~\ref{fig-unld-melt-ratio}).

When canopy snow loads remained between 0 and 5 mm, the
unloading-to-melt ratio varied from approximately 0 to 0.5. As snow
loads increased, this ratio increased linearly, reaching its peak value
of 5.0 for a canopy snow load of 30 mm
(Figure~\ref{fig-unld-melt-ratio}). This relationship can be expressed
through a linear function:

\begin{equation}\phantomsection\label{eq-q-unld-melt}{
q_{unld}^{melt} = R \cdot q_{melt}(L)
}\end{equation}

where \(R\) represents the unloading-to-snowmelt ratio and \(q_{melt}\)
is the canopy snowmelt rate (mm hr\textsuperscript{-1}).
Equation~\ref{eq-q-unld-melt} is similar to Equation 33 in Andreadis et
al. (2009); however, instead of a constant value of 0.4 for \(R\), it
was determined as:

\begin{equation}\phantomsection\label{eq-fm}{
R = m \cdot L + b
}\end{equation}

where \(m\) and \(b\) were determined as 0.16 and -0.5, respectively,
using an ordinary least squares regression.
Equation~\ref{eq-q-unld-melt} using these fitted parameters resulted in
a statistically significant relationship (\emph{p} \textless{} 0.05,
\emph{R\textsuperscript{2}} = 0.73) (Figure~\ref{fig-unld-melt-ratio}).
Additional observations of canopy snowmelt from the subcanopy rain
gauges were also used to estimate \(R\)
(Figure~\ref{fig-unld-melt-ratio}). The number of usable observations
was limited to three events (out of the 5 warm \& humid events) due to
freeze-thaw events that seized up the tipping bucket mechanisms in the
rain gauges, however, these measurements are still useful in providing
some validation of the CRHM canopy snowmelt/drip calculations
(Figure~\ref{fig-cml-tb}). The CRHM-estimated cumulative drip was higher
than the subcanopy rain gauges for two out of the three melt events.
Differences in the timing and magnitude of the observed and simulated
values were expected due to both instrument uncertainties in the rain
gauges from freezing of rain gauges and thawing of snow in the
collection funnels, and in the canopy snow energy balance simulation.

\begin{figure}[htbp]

\centering{

\includegraphics[width=0.8\linewidth,height=\textheight,keepaspectratio]{figs/final/figure6.png}

}

\caption{\label{fig-unld-melt-ratio}The ratio of canopy snow unloading
(weighed tree residual) to snowmelt across different canopy snow load
bins and events. Black dots represent the observed cumulative unloading
divided by the cumulative simulated snowmelt from the updated CP25
canopy snow routine in CRHM for each of the five warm \& humid events.
Red dots show the cumulative observed unloading divided by snowmelt
measured by the rain gauges. Multiple dots within a bin correspond to
different events. The blue line represents the best-fit line derived
from ordinary least squares regression.}

\end{figure}%

\begin{figure}[htbp]

\centering{

\includegraphics[width=0.85\linewidth,height=\textheight,keepaspectratio]{figs/final/figure7.png}

}

\caption{\label{fig-cml-tb}Cumulative canopy snow drip measured by the
average of four subcanopy tipping bucket rain gauges (TB) and simulated
using the CRHM CP25 model (Equation~\ref{eq-eb}). Yellow shading
indicates the range of ±1 standard deviation amongst the individual rain
gauge measurements.}

\end{figure}%

\subsection{New Canopy Snow Model}\label{new-canopy-snow-model}

The CP25 model is based on the canopy snow mass balance formulation
(Equation~\ref{eq-canopy-mass-bal}), where \(q_{tf}\) was represented
as:

\begin{equation}\phantomsection\label{eq-qtf}{
q_{tf} = [(1 - C_p)  \alpha] q_{sf}
}\end{equation}

where \(C_p\) is the leaf contact area calculated using Equation 10 in
Cebulski \& Pomeroy (2025b) and \(\alpha\) is an efficiency constant.
Melt-driven unloading (\(q_{unld}^{melt}\)) was modelled using
Equation~\ref{eq-q-unld-melt}, while dry snow unloading
(\(q_{unld}^{dry}\)) was represented using Equation~\ref{eq-q-unld-tau}.
Canopy snow drip (\(q_{drip}\)) is derived from calculations of canopy
snowmelt from Equation~\ref{eq-eb}, with storage limited by the canopy
liquid water holding capacity computed from
Equation~\ref{eq-liq-hold-cap}; any excess was assumed to immediately
drain. Wind transport of canopy snow (\(q_{wind}^{veg}\)) is
incorporated in the Equation~\ref{eq-q-unld-tau} calculation.
Sublimation of intercepted snow (\(q_{veg}^{sub}\)) was represented
using Equation~\ref{eq-ql}.

\subsection{Event-based Evaluation of Canopy Snow Ablation
Models}\label{event-based-evaluation-of-canopy-snow-ablation-models}

The updated canopy snow model (CP25), as well as the existing models
R01, SA09, and E10 were evaluated using weighed tree lysimeter
measurements of canopy load during seventeen canopy snow ablation
events. The seventeen ablation events had air temperatures ranging from
-30.5°C to +6.9°C and wind speeds from calm to 5.3 m
s\textsuperscript{-1} (Figure~\ref{fig-event-met-boxplot}). This range
of meteorological conditions is particularly wide and spans conditions
typical of the cold boreal to temperate maritime needleleaf forests.
Events were classified as cold \& dry, cold \& humid, warm \& dry, and
warm \& humid based on the median event air temperature (above or below
0°C) and relative humidity (above or below 65\%)
(Figure~\ref{fig-event-met-boxplot}).

\begin{figure}[htbp]

\centering{

\includegraphics[width=\linewidth,height=0.65\textheight,keepaspectratio]{figs/final/figure8.png}

}

\caption{\label{fig-event-met-boxplot}Boxplots showing the distribution
of meteorological measurements of air temperature, relative humidity,
and wind speed over each of the seventeen select ablation events. Air
temperature, relative humidity, and wind speed were measured at FT
station. Note: the rectangle vertical extent represents the
interquartile range (25\textsuperscript{th} to 75\textsuperscript{th}
percentile), the horizontal line within each box indicates the median,
and the whiskers extend to 1.5 times the interquartile range. Circular
points beyond the whiskers represent outliers.}

\end{figure}%

Simulated canopy snow load by the CP25 model closely matched the
observations for all 17 events, demonstrating the most consistent
agreement amongst the models evaluated
(Figure~\ref{fig-obs-mod-w-tree}). The large declines in canopy snow
load for E10 (Figure~\ref{fig-obs-mod-w-tree}) are due to the maximum
canopy snow load used in this model which ranged from 7 to 12 mm
depending on the fresh snow density (a function of air temperature in
Hedstrom \& Pomeroy, 1998).

The energy balance-based snowmelt modelling approaches (CP25 \& SA09)
yielded the most accurate representation of canopy snowmelt over the
warm \& humid events. The CP25 mean bias of 0.02 mm
hr\textsuperscript{-1}, which was smaller than the -0.11 mm
hr\textsuperscript{-1} bias associated with SA09
(Figure~\ref{fig-cpy-load-mb-boxplot}). The improvement for CP25 over
SA09 comes from its representation of the increase in unloading at
higher canopy snow loads (Figure~\ref{fig-unld-melt-ratio}), as observed
for the 2022-06-14 event (Figure~\ref{fig-obs-mod-w-tree}). The air
temperature (R01) and ice-bulb temperature (E10) models produced a wider
range of mean biases (Figure~\ref{fig-cpy-load-mb-boxplot}) and larger
mean bias of over 0.42 mm hr\textsuperscript{-1} during the warm \&
humid events, compared to CP25 and SA09. The rate of ablation was slower
for canopy snow loads below \textasciitilde1.5 mm and \textasciitilde0.3
mm for CP25 and SA09 respectively; due to their differing liquid water
storage capacities (Figure~\ref{fig-obs-mod-w-tree}). For the warm
events other than 2022-04-23, the observed decline in ablation rate
occurs around 2 to 3 mm, exceeding the threshold predicted by all
models.

The warm \& dry events had consistent performance with mean biases of
0.02 mm hr\textsuperscript{-1} for each of the four models
(Figure~\ref{fig-cpy-load-mb-boxplot}). Initiation of ablation was
delayed compared to observations from the weighed tree for the CP25 and
SA09 models for all three of the warm \& dry events
(Figure~\ref{fig-obs-mod-w-tree}). The temperature threshold methods
(E10 \& R01) achieved better timing on the onset of ablation for two
events (2022-03-29 and 2022-04-21) compared to CP25. However, E10
initiated the onset of ablation slightly earlier for 2022-04-23 and the
rate of ablation was also lower than observed for this event.

\begin{figure}[htbp]

\centering{

\includegraphics[width=1\linewidth,height=\textheight,keepaspectratio]{figs/final/figure9.png}

}

\caption{\label{fig-obs-mod-w-tree}Time series of canopy snow load for
individual events measured by the weighed tree (observed) and simulated
using the four canopy snow models.}

\end{figure}%

For the cold \& dry events, all models had accurate performance with
mean biases ranging from -0.01 to 0.01 mm hr\textsuperscript{-1}, with a
slight improvement in the mean bias for CP25 of -0.009 mm
hr\textsuperscript{-1} (Figure~\ref{fig-cpy-load-mb-boxplot}). Canopy
snowmelt was overestimated by the E10 model and caused the steep initial
decline in canopy snow load on 2022-03-02 which is not registered by the
weighed tree or other models. However, underestimates of ablation
compensated for this overestimate over the remaining cold \& dry
events---which had moderate wind speeds---as wind-driven unloading is
not included in the E10 model (Figure~\ref{fig-event-met-boxplot}). For
events 2022-03-02 and 2023-03-14 the R01 model overestimated canopy snow
ablation due to an overestimation of wind-driven unloading
(Figure~\ref{fig-obs-mod-w-tree}).

The importance of representing wind-driven unloading was clear during
the cold \& humid events, where the mean bias of models including this
mechanism was reduced compared to other approaches; for example, 0.04 mm
hr\textsuperscript{-1} for R01 and 0.15 mm hr\textsuperscript{-1} for
CP25. In contrast, simulations that did not explicitly account for
wind-driven unloading exhibited higher biases, exceeding 0.32 mm
hr\textsuperscript{-1} (Figure~\ref{fig-cpy-load-mb-boxplot}). Although
the E10 model does not include wind-driven unloading, it performed best
for the 2023-02-26 event due to its relatively slow time-based unloading
rate compared to CP25 and R01 which overestimate ablation for this event
(Figure~\ref{fig-obs-mod-w-tree}). The R01 model overestimated unloading
over most of the cold \& humid events and had a higher median bias for
the cold \& humid events compared to CP25
(Figure~\ref{fig-cpy-load-mb-boxplot}). The CP25 model had consistently
lower bias across the three cold \& humid events, but still
underestimated ablation for the 2023-02-24 event which had peak wind
speeds of over 5 m s\textsuperscript{-1}. Over this event, 1.3 mm of
snow was measured at a shielded precipitation gauge in a nearby clearing
and was likely derived from wind transport of snow from the canopy, as
clear skies with no precipitation were observed. The amount of snow
observed to unload from the canopy into the subcanopy lysimeters during
this event was consistent with simulated unloading in CRHM suggesting
that the remaining unaccounted-for snow was likely entrained into the
atmosphere and sublimated and/or transported to distant sites.

\begin{figure}[htbp]

\centering{

\pandocbounded{\includegraphics[keepaspectratio]{figs/final/figure10.png}}

}

\caption{\label{fig-cpy-load-mb-boxplot}Boxplots illustrating the
distribution of event mean biases calculated between simulations of
canopy snowload and observations from the weighed tree. The vertical
extent of each rectangle represents the interquartile range
(25\textsuperscript{th} to 75\textsuperscript{th} percentile), the
horizontal line within each box indicates the median, and the whiskers
extend to 1.5 times the interquartile range. Circular points beyond the
whiskers represent outliers. The diamonds represent the mean of the
event biases.}

\end{figure}%

\subsection{Canopy Snow Partitioning}\label{canopy-snow-partitioning}

During warm \& humid events, all four parameterisations showed
relatively consistent partitioning of canopy snow, with only a small
fraction returned to the atmosphere (Figure~\ref{fig-partitioning}). The
warm \& dry events had greater variability in the partitioning of
intercepted snow---when compared to the warm \& humid events---with a
larger contribution from sublimation and evaporation processes.
Increased unloading from the E10 model resulted in a greater fraction of
intercepted snow reaching the ground compared to the CP25 model over the
warm \& dry events.

For the cold \& dry and cold \& humid events, the two parameterisations
that include wind-driven unloading (R01 and CP25) had differing
fractions of snow partitioned to the ground via unloading. A higher
fraction of intercepted snow reached the ground for R01 due to the
higher dry snow unloading rate compared to CP25 over all of the cold
events. For both the cold \& dry and cold \& humid events SA09
partitioned all snow back to the atmosphere via sublimation as dry snow
unloading is not included in this parameterisation. Although CP25 and
E10 include differing unloading processes, when averaged over all
events, both had a similar fraction of snow reaching the ground (70\%)
versus the atmosphere (30\%) (Table~\ref{tbl-frac-atm-ground}). SA09 has
the largest discrepancy which returned 40\% of intercepted snow back to
the atmosphere and the R01 with the least amount of snow reaching the
atmosphere with 24\%.

\begin{figure}[htbp]

\centering{

\includegraphics[width=1\linewidth,height=\textheight,keepaspectratio]{figs/final/figure11.png}

}

\caption{\label{fig-partitioning}Bar chart illustrating the proportion
of intercepted snow that was either lost to the atmosphere as
sublimation and/or evaporation of melted snow or transferred to the
ground through unloading or drip of melted snow by each event type for
all 17 events.}

\end{figure}%

\begin{longtable}[]{@{}lrr@{}}

\caption{\label{tbl-frac-atm-ground}Fraction of canopy-intercepted snow
returned to the atmosphere as sublimation and evaporation of melted snow
or input to the ground as unloading or drip of melted snow for each
parameterisation over the 17 select ablation events.}

\tabularnewline

\toprule\noalign{}
Model & Atmosphere (-) & Ground (-) \\
\midrule\noalign{}
\endhead
\bottomrule\noalign{}
\endlastfoot
CP25 & 0.29 & 0.71 \\
E10 & 0.31 & 0.69 \\
R01 & 0.24 & 0.76 \\
SA09 & 0.40 & 0.60 \\

\end{longtable}

\section{Discussion}\label{discussion}

\subsection{Processes Governing Canopy Snow
Unloading}\label{processes-governing-canopy-snow-unloading}

Observations of canopy snow unloading support the hypothesis that
unloading is primarily controlled by snowmelt and dry-snow related
processes, both of which are also influenced by the amount of snow
intercepted in the canopy (Table~\ref{tbl-q-unld-bins}). The ratio of
unloading to canopy snowmelt was found to increase linearly with
increasing canopy snow load (Figure~\ref{fig-unld-melt-ratio}), which
differs from Storck et al. (2002) who originally found the ratio of
canopy snow unloading to melt to be constant at 0.4. The measurement
difficulties noted by Storck et al. (2002) limited their estimate of
this ratio to a single mid-December event, preventing any association
with canopy snow load. Similar instrument difficulties here in measuring
canopy snowmelt drainage limited direct measurements to three events and
hybrid measurements (from Equation~\ref{eq-unld-melt-mass-bal}) to 5
events (Figure~\ref{fig-unld-melt-ratio}). The reasonable correspondence
between observed and modelled canopy snow drip for these events supports
the linear increase hypothesis (Figure~\ref{fig-cml-tb}). Previous
studies have identified relationships between melt-induced unloading and
various meteorological parameters, including empirical functions of air
temperature (Katsushima et al., 2023; Roesch et al., 2001), ice-bulb
temperature (Ellis et al., 2010; Floyd, 2012), and solar radiation
(Katsushima et al., 2023). Although branch bending and subsequent
unloading has been shown to be associated with air temperature (Schmidt
\& Gluns, 1991; Schmidt \& Pomeroy, 1990), observations here indicate
that temperature-related unloading increases occur primarily near the
freezing point (Figure~\ref{fig-q-unld-all-bins}), eliminating the need
for a separate temperature parameterisation beyond snowmelt-associated
unloading processes.

Dry snow unloading was found to increase exponentially with wind speed
and linearly with shear stress (Table~\ref{tbl-q-unld-wind}). These
results differ from earlier research that has represented this process
as a linear function of wind speed and canopy load (Bartlett \&
Verseghy, 2015; Katsushima et al., 2023; Roesch et al., 2001), as shown
in Figure~\ref{fig-unl-ex-wind}. The higher \emph{R\textsuperscript{2}}
found for shear stress compared to wind speed for predicting
unloading---when excluding melt events---is likely due to the physical
relationship between shear stress force and kinetic energy transfer to
the canopy, wind transport from the canopy, and movement of branches in
the canopy induced by drag (shear) forces. The differing relationship
presented here, compared to the R01 model
(Figure~\ref{fig-unl-ex-wind}), may be attributed to the development of
that parameterisation using above canopy albedo as a proxy for canopy
snow unloading (Bartlett \& Verseghy, 2015; Roesch et al., 2001). This
approach would have included both unloading and sublimation processes,
in addition to greater measurement uncertainties (Cebulski \& Pomeroy,
2025a). Conversely, the subcanopy lysimeter measurements employed here
provided a more direct quantification of canopy snow unloading rates.
Simulated unloading over events classified as cold \& humid---which had
the largest contribution of wind-driven unloading---resulted in the
highest overall mean biases compared to the warm \& dry, warm \& humid,
and cold \& dry events (Figure~\ref{fig-cpy-load-mb-boxplot}).
Additional factors that may influence dry snow unloading that are not
considered in the new parameterisation (Equation~\ref{eq-q-unld-wind})
include wind erosion, branch movement, structural degradation, bond
weakening, increased elasticity of branches, snow density, and liquid
water content. The addition of liquid water content in the canopy snow
due to phase change can increase cohesion and adhesion of snow clumps
within the canopy (Pomeroy \& Gray, 1995). However, high liquid water
contents during rapid melt can lubricate the snow attachment to the
canopy and weaken cohesive bonds, inducing unloading, much as for wet
snow avalanches (Baggi \& Schweizer, 2008).

The density of snow intercepted in the canopy is expected to influence
both dry snow and melt-induced unloading processes---and is incorporated
in the E10 parameterisation for the initial accumulation component based
on the findings of Schmidt \& Gluns (1991)---but is not explicitly
represented in any of the ablation calculations included in this study.
Fresh, low-density snow typically exhibits lower cohesion and adhesion
compared to older snow, which may have undergone freeze-thaw cycles or
equitemperature metamorphism, processes that increase snow density and
bond strength hence, increasing the mechanical resistance to unloading.
While vapour deposition and rime-ice accumulation are simulated in some
models (e.g., Clark et al., 2015; Ellis et al., 2010) via the latent
heat flux parameterisation, they are usually treated as additions to the
canopy snow reservoir. However, in humid or maritime regions rime can
form dense, ice-like structures (e.g., Berndt \& Fowler, 1969) with high
resistance to unloading by either melt or wind (Lumbrazo et al., 2022).
Although canopy snow density is expected to influence ablation
processes, it was not observed in this study due to its continental
location and remains a research gap for maritime climates.

The relationship between unloading and canopy snow sublimation was not
statistically significant (Figure~\ref{fig-q-unld-all-bins}). This
differs from earlier work by MacDonald (2010) who found an association
between these two variables and attributed this to the reduction in
structural integrity and bond weakening of the canopy snow clumps as
snow particles are removed through sublimation. It is possible that the
association identified by MacDonald (2010) arose from the concurrent
increase in canopy energy inputs that promote both sublimation and other
ablation mechanisms such as melt and other dry snow unloading processes
which were not directly accounted for.

\subsection{Performance Comparison of Ablation
Models}\label{performance-comparison-of-ablation-models}

The improved performance of the new CP25 model across a wide range of
meteorological conditions demonstrates the advantages of incorporating
comprehensive snow unloading processes coupled with physically based
representations of the energy balance to simulate ablation
(Figure~\ref{fig-obs-mod-w-tree}). In contrast, existing models were
limited by missing processes such as wind-driven unloading (SA09 and
E10) or relied on temperature-dependent parameterisations of melt and
drip processes which had limited transferability across differing events
(R01 and E10). Although the SA09 model---originally developed in a
relatively warm maritime climate with limited wind influence (Storck et
al., 2002)---performed similarly well to CP25 during most of the
melt-dominated events, its exclusion of wind-induced unloading led to
poor performance for the cold \& dry and cold \& humid events. This
process omission caused SA09 to overestimate sublimation when averaged
over all events (Table~\ref{tbl-frac-atm-ground}). Moreover, the
constant canopy snow unloading to melt ratio of 0.4 in the SA09 model
led to reduced performance for one melt event with canopy snow loads up
to 30 mm, whereas the CP25 model more accurately predicted ablation over
this event. The low liquid water retention capacity implemented in SA09
also contributed to an underestimation of canopy loads during the end of
most melt events (Figure~\ref{fig-obs-mod-w-tree}).

Models utilising temperature-based canopy snowmelt parameterisations
(E10 and R01) exhibited inconsistent performance, particularly during
warm \& humid events where they underestimated ablation. This was due to
their reliance on air temperature or ice-bulb temperature as proxies for
energy input into the canopy, which failed to represent the energy
availability over these events. In contrast, all four models performed
similarly during warm \& dry events, where air temperatures were closer
to 0°C. The superior performance of the energy balance-based canopy
snowmelt models (CP25 and SA09) during warm \& humid events likely
reflects their ability to represent the elevated energy inputs which
were present over the warmer events.

While E10 omitted any explicit representation of wind-driven unloading,
its exponential time decay parameterisation indirectly addressed this
process, though it still underestimated overall ablation for the cold
events which had higher wind speeds (Figure~\ref{fig-obs-mod-w-tree}).
The maximum canopy snow load threshold implemented in the E10 model was
lower than observations from the weighed tree. This limitation offset
its tendency to underestimate unloading processes as was also observed
by Lundquist et al. (2021) and Lumbrazo et al. (2022). In this study,
the CP25, SA09, and R01 models did not include a maximum snow load, and
their performance aligns with the hypothesis of Lundquist et al. (2021)
that this limit may be unnecessary---or much higher than previously
though---when combined with a comprehensive canopy snow ablation
routine. In contrast, R01 consistently overestimated wind-driven
unloading during both the cold \& dry and cold \& humid events,
potentially due to the differing methodology used to develop this
parameterisation (Figure~\ref{fig-obs-mod-w-tree}). CP25 provided a
better representation of wind-driven unloading compared to R01 aside
from one one event with high wind speeds (\textgreater5 m
s\textsuperscript{-1})---that cause wind redistribution/entrainment of
snow into the atmosphere. Wind unloading parameters also may be
influenced by tree species, forest structure, and snow characteristics
(Lumbrazo et al., 2022), necessitating further field-based research on
canopy snow unloading in diverse environments to assess their broader
applicability.

The infrequent occurrence of wind-transport from the canopy in our
observations may account for underestimation of ablation during one
strongly wind-dominated events by CP25 (2023-02-24,
Figure~\ref{fig-obs-mod-w-tree}). Wind transport of canopy snow to the
nearby Powerline snowfall gauge occurred during this event, but was a
small fraction of canopy snow ablation, 1.3 mm of a total of 20 mm of
canopy snow ablation. Since unloading measured by the subcanopy
lysimeters corresponded well with CP25 predictions for this event, the
approximately 9 mm of unaccounted canopy snow ablation may be attributed
to uncertainties within the wind-driven unloading parameterisation
(Figure~\ref{fig-q-unld-wind}) and possible atmospheric entrainment of
canopy snow that was either transported to distant locations and/or
sublimated. These findings align with observations by Troendle (1983)
but contrast with Hoover \& Leaf (1967), who proposed that most
wind-transported snow relocates to nearby sites with minimal sublimation
effects.

\subsection{Canopy Snow Partitioning}\label{canopy-snow-partitioning-1}

Substantial variability was found in the fraction of snow that
sublimated and/or evaporated as liquid meltwater versus unloading and
drip depending on the canopy snow ablation model selected
(Figure~\ref{fig-partitioning}). For example, the exclusion of
wind-driven unloading processes in the SA09 model resulted in 100\% of
intercepted snow reaching the atmosphere for both the cold \& dry and
cold \& humid events. This differed considerably from the CP25 model
which returned 70\% and 24\% of snow back to the atmosphere for the cold
\& dry and cold \& humid events, respectively. Although E10 and CP25
include differing process representations, they predicted comparable
fractions of snow reaching the ground versus returning to the
atmosphere. The agreement between CP25 and E10 is notable, since the E10
has been tested and found to perform well in predicting subcanopy
snowpacks around the world (Ellis et al., 2010; Gelfan et al., 2004;
Pomeroy et al., 2022; Sanmiguel-Vallelado et al., 2022). However, the
results shown here reveal that E10's individual process representations
can be in error, particularly under warm and windy conditions,
potentially explaining the difficulties when applying E10 at locations
where parameterisation errors fail to offset one another (Lumbrazo et
al., 2022; Lundquist et al., 2021). For example, at locations which
intercept a larger amount of snow, the E10 maximum canopy snow load
would overestimate the amount of unloading, and a greater deviation
between the E10 and CP25 model is expected.

\subsection{Future Directions}\label{future-directions}

Physically based approaches such as CP25 are particularly relevant for
predictions of snow hydrology under a changing climate, where warming
may reduce the reliability of empirically derived canopy snowmelt models
like E10 and R01. The improved representation of melt events by CP25 and
SA09 demonstrates the reliability of more physically based methods
across a range of meteorological conditions, compared to
temperature-based canopy snowmelt routines (E10 and R01) which had
reduced performance over these events. Amongst all canopy snow ablation
processes, dry snow unloading introduced the most uncertainty. Although
the revised model performed best for this dataset, further validation is
required across a wider range of climates and forest structures. Since
unloading, melt, and sublimation are competitive ablation processes,
they strongly influence whether snow is returned to the atmosphere or
reaches the ground.

Key limitations remain in measuring canopy snow sublimation using eddy
correlation systems (Conway et al., 2018; Harding \& Pomeroy, 1996;
Harvey et al., 2025; Helgason \& Pomeroy, 2012b; Parviainen \& Pomeroy,
2000) and separating snow unloading from meltwater drip (Floyd, 2012;
Storck et al., 2002), which limit the development and testing of canopy
snow ablation parameterisations. Whilst separating initial interception
and ablation processes (Cebulski \& Pomeroy, 2025a; Cebulski \& Pomeroy,
2025b), will improve process representations, these routines still need
to be evaluated together against additional field observations.
Incorporating the updated unloading schemes developed here could improve
the representation of canopy snow ablation and, by extension, the
partitioning of precipitation and canopy albedo in hydrological and land
surface models. Nonetheless, further testing is needed across different
sites, climates, forest types, and spatial scales to assess model
transferability and performance.

\section{Conclusions}\label{conclusions}

Canopy snow ablation processes govern the timing and partitioning of
snowfall to the ground versus the atmosphere in forested environments,
yet their representation in modelling frameworks remains uncertain due
to insufficient process-level validation. This study evaluates existing
canopy snow ablation theories using in-situ measurements of canopy snow
load, unloading, and drip combined with a novel canopy snow energy and
mass balance model. These observations revealed that canopy snow load,
wind shear stress, and canopy snowmelt were statistically significant
predictors of snow unloading, collectively explaining 80\% of its
variability. In addition to this empirical evidence, physical processes
such as structural degradation, snow particle bond weakening,
lubrication of wet canopy snow during melt, and the shear force exerted
on canopy snow by wind further support representing these processes.
Although some studies use air temperature as an index of unloading
resulting from canopy snowmelt and potential branch bending, here energy
balance methodologies show improved performance in simulating canopy
load during melt events.

Previous studies have demonstrated relationships between unloading and
snow load, wind speed, and canopy snowmelt rate, but these processes
have not been evaluated collectively. This study represents the first
development and validation of an unloading model addressing both energy
balance-based melt and dry snow unloading processes together. Novel
parameterisations for dry snow and melt-induced unloading were
introduced, with key differences from previously established approaches.
Shear stress was found to be a stronger predictor of dry snow unloading
(\emph{R\textsuperscript{2}} = 0.61) than wind speed
(\emph{R\textsuperscript{2}} = 0.54) for non-melt periods. The canopy
melt rate exerted the strongest control on snow unloading during melt
events, consistent with one existing model. A new finding was that the
ratio of unloading to canopy snowmelt increased with canopy snow load.
Additionally, an existing approach which used the concept of a maximum
intercepted snow load greatly underestimated the canopy snow storage
capacity when compared to observed snow loads from weighed tree
measurements. Throughout the two-years of observations presented here, a
maximum canopy snow load was not observed, likely as unloading rates
increased with higher snow loads. Wind transport events were relatively
rare in this wind-exposed subalpine forest, but resulted in a
considerable underestimation of the amount of snow returned to the
atmosphere or surrounding sites during one event.

A new canopy snow ablation model that integrates an updated canopy snow
mass and energy balance demonstrated improved accuracy across varied
meteorological conditions compared to existing approaches. Existing
models failed to maintain accuracy across events with a wide range of
meteorology due to neglect of key processes and/or empirical
representations of melt processes. The greatest inter-model
discrepancies in canopy snow load occurred during warm and humid events,
where temperature-based canopy snowmelt parameterisations showed
substantially higher mean biases relative to energy balance-based
models.

Amongst the models tested, the largest errors were found during cold \&
dry unloading events---though performance was improved when
incorporating a site-specific shear stress-based parameterisation.
Partitioning of intercepted snow disposition between the ground and
atmosphere varied most amongst cold events, where neglecting the dry
snow unloading process resulted in considerable overestimates of canopy
snow sublimation losses. All canopy snow models had greater consistency
in partitioning canopy snow during warm \& humid events, where all
canopy snow was typically unloaded or melted as drip towards the ground
surface. However, the rate of unloading was best represented by energy
balance-based canopy snowmelt routines compared to empirical
relationships. Although improved performance was found for the updated
canopy snow ablation model compared to existing methods, across a wide
range of meteorological conditions, additional testing across various
climate and forest compositions is required to assess model
transferability.

\section{Acknowledgements}\label{acknowledgements}

We acknowledge financial support from the University of Saskatchewan
Dean's Scholarship, the Natural Sciences and Engineering Research
Council of Canada's Discovery Grants, the Canada First Research
Excellence Fund's Global Water Futures Programme, Environment and
Climate Change Canada, Alberta Innovates Water Innovation Program, the
Canada Foundation for Innovation's Global Water Futures Observatories
facility, and the Canada Research Chairs Programme. We thank Madison
Harasyn, Hannah Koslowsky, Kieran Lehan, Lindsey Langs and Fortress
Mountain Resort for their help in the field and Tom Brown and Logan Fang
for support of the CRHM platform.

\section{Data \& Software Availability
Statement}\label{data-software-availability-statement}

The Cold Regions Hydrological Model Platform (CRHM) source code is
available at https://github.com/srlabUsask/crhmcode. Model forcing data,
model outputs, validation data, processed data, and scripts are
available at https://doi.org/10.5281/zenodo.16898881.

\pagebreak

\section*{References}\label{references}
\addcontentsline{toc}{section}{References}

\phantomsection\label{refs}
\begin{CSLReferences}{1}{0}
\bibitem[\citeproctext]{ref-Allan1998}
Allan, R., Pereira, L., Raes, D., \& Smith, M. (1998). \emph{Crop
evapotranspiration {Guidelines} for computing crop water requirements}.
{Food and Agriculture Organization of the United Nations}.

\bibitem[\citeproctext]{ref-Andreadis2009}
Andreadis, K. M., Storck, P., \& Lettenmaier, D. P. (2009). Modeling
snow accumulation and ablation processes in forested environments.
\emph{Water Resources Research}, \emph{45}(5), 1--33.
\url{https://doi.org/10.1029/2008WR007042}

\bibitem[\citeproctext]{ref-Aubry-Wake2023}
Aubry-Wake, C., \& Pomeroy, J. W. (2023). Predicting hydrological change
in an alpine glacierized basin and its sensitivity to landscape
evolution and meteorological forcings. \emph{Water Resources Research},
\emph{59}(9). \url{https://doi.org/10.1029/2022WR033363}

\bibitem[\citeproctext]{ref-Baggi2008}
Baggi, S., \& Schweizer, J. (2008). Characteristics of wet-snow
avalanche activity: 20 years of observations from a high alpine valley
({Dischma}, {Switzerland}). \emph{Natural Hazards}, \emph{50}(1),
97--108. \url{https://doi.org/10.1007/s11069-008-9322-7}

\bibitem[\citeproctext]{ref-Bartlett2015}
Bartlett, P. A., \& Verseghy, D. L. (2015). Modified treatment of
intercepted snow improves the simulated forest albedo in the {Canadian
Land Surface Scheme}. \emph{Hydrological Processes}, \emph{29}(14),
3208--3226. \url{https://doi.org/10.1002/HYP.10431}

\bibitem[\citeproctext]{ref-Berndt1969}
Berndt, H. W., \& Fowler, W. B. (1969). Rime and hoarfrost in
upper-slope forests of eastern washington. \emph{Journal of Forestry},
\emph{67}(2), 92--95. \url{https://doi.org/10.1093/jof/67.2.92}

\bibitem[\citeproctext]{ref-Cebulski2025}
Cebulski, A. C., \& Pomeroy, J. W. (2025a). Theoretical {Underpinnings}
of {Snow Interception} and {Canopy Snow Ablation Parameterisations}.
\emph{WIREs Water}, \emph{12}, e70010.
\url{https://doi.org/10.1002/wat2.70010}

\bibitem[\citeproctext]{ref-Cebulski2025a}
Cebulski, A. C., \& Pomeroy, J. W. (2025b). Snow {Interception
Relationships With Meteorology} and {Canopy Density}. \emph{Hydrological
Processes}, \emph{39}(4), e70135.
\url{https://doi.org/10.1002/hyp.70135}

\bibitem[\citeproctext]{ref-Clark2015b}
Clark, M. P., Nijssen, B., Lundquist, J. D., Kavetski, D., Rupp, D. E.,
Woods, R. A., Freer, J. E., Gutmann, E. D., Wood, A. W., Gochis, D. J.,
Rasmussen, R. M., Tarboton, D. G., Mahat, V., Flerchinger, G. N., \&
Marks, D. G. (2015). A unified approach for process-based hydrologic
modeling: 2. {Model} implementation and case studies. \emph{Water
Resources Research}, \emph{51}(4), 2515--2542.
\url{https://doi.org/10.1002/2015WR017200}

\bibitem[\citeproctext]{ref-Conway2018}
Conway, J. P., Pomeroy, J. W., Helgason, W. D., \& Kinar, N. J. (2018).
Challenges in modeling turbulent heat fluxes to snowpacks in forest
clearings. \emph{Journal of Hydrometeorology}, \emph{19}(10),
1599--1616. \url{https://doi.org/10.1175/JHM-D-18-0050.1}

\bibitem[\citeproctext]{ref-Ellis2010}
Ellis, C. R., Pomeroy, J. W., Brown, T., \& MacDonald, J. (2010).
Simulation of snow accumulation and melt in needleleaf forest
environments. \emph{Hydrology and Earth System Sciences}, \emph{14}(6),
925--940. \url{https://doi.org/10.5194/hess-14-925-2010}

\bibitem[\citeproctext]{ref-Essery2003}
Essery, R., Pomeroy, J. W., Parviainen, J., \& Storck, P. (2003).
Sublimation of snow from coniferous forests in a climate model.
\emph{Journal of Climate}, \emph{16}(11), 1855--1864.
\url{https://doi.org/10.1175/1520-0442(2003)016\%3C1855:SOSFCF\%3E2.0.CO;2}

\bibitem[\citeproctext]{ref-Fang2023}
Fang, X., \& Pomeroy, J. W. (2023). Simulation of the impact of future
changes in climate on the hydrology of {Bow River} headwater basins in
the {Canadian Rockies}. \emph{Journal of Hydrology}, \emph{620}, 129566.
\url{https://doi.org/10.1016/j.jhydrol.2023.129566}

\bibitem[\citeproctext]{ref-Floyd2012}
Floyd, W. C. (2012). \emph{Snowmelt energy flux recovery during
rain-on-snow in regenerating forests} (p. 180) {[}PhD thesis, University
of British Columbia{]}.
https://doi.org/\url{https://dx.doi.org/10.14288/1.0073024}

\bibitem[\citeproctext]{ref-Gelfan2004}
Gelfan, A. N., Pomeroy, J. W., \& Kuchment, L. S. (2004). Modeling
forest cover influences on snow accumulation, sublimation, and melt.
\emph{Journal of Hydrometeorology}, \emph{5}(5), 785--803.
\url{https://doi.org/10.1175/1525-7541(2004)005\%3C0785:MFCIOS\%3E2.0.CO;2}

\bibitem[\citeproctext]{ref-Harder2013}
Harder, P., \& Pomeroy, J. W. (2013). Estimating precipitation phase
using a psychrometric energy balance method. \emph{Hydrological
Processes}, \emph{27}(13), 1901--1914.
\url{https://doi.org/10.1002/hyp.9799}

\bibitem[\citeproctext]{ref-Harding1996}
Harding, R. J., \& Pomeroy, J. W. (1996). The {Energy Balance} of the
{Winter Boreal Landscape}. \emph{Journal of Climate}, \emph{9}(11),
2778--2787. \url{https://www.jstor.org/stable/26201420}

\bibitem[\citeproctext]{ref-Harvey2025}
Harvey, N., Burns, S. P., Musselman, K. N., Barnard, H., \& Blanken, P.
D. (2025). Identifying {Canopy Snow} in {Subalpine Forests}: {A
Comparative Study} of {Methods}. \emph{Water Resources Research},
\emph{61}(1), e2023WR036996. \url{https://doi.org/10.1029/2023WR036996}

\bibitem[\citeproctext]{ref-Hedstrom1998}
Hedstrom, N. R., \& Pomeroy, J. W. (1998). Measurements and modelling of
snow interception in the boreal forest. \emph{Hydrological Processes},
\emph{12}(10-11), 1611--1625.
\url{https://doi.org/10.1002/(SICI)1099-1085(199808/09)12:10/11\%3C1611::AID-HYP684\%3E3.0.CO;2-4}

\bibitem[\citeproctext]{ref-Helgason2012a}
Helgason, W., \& Pomeroy, J. W. (2012a). Characteristics of the
near-surface boundary layer within a mountain valley during winter.
\emph{Journal of Applied Meteorology and Climatology}, \emph{51}(3),
583--597. \url{https://doi.org/10.1175/JAMC-D-11-058.1}

\bibitem[\citeproctext]{ref-Helgason2012b}
Helgason, W., \& Pomeroy, J. W. (2012b). Problems closing the energy
balance over a homogeneous snow cover during midwinter. \emph{Journal of
Hydrometeorology}, \emph{13}(2), 557--572.
\url{https://doi.org/10.1175/JHM-D-11-0135.1}

\bibitem[\citeproctext]{ref-Hoover1967}
Hoover, M. D., \& Leaf, C. F. (1967). Processs and {Significance} of
{Interception} in {Colorado Subalpine Forest}. \emph{Proceeding of a
{National Science Foundation Advanced Science Seminar}}.

\bibitem[\citeproctext]{ref-Immerzeel2020}
Immerzeel, W. W., Lutz, A. F., Andrade, M., Bahl, A., Biemans, H.,
Bolch, T., Hyde, S., Brumby, S., Davies, B. J., Elmore, A. C., Emmer,
A., Feng, M., Fernández, A., Haritashya, U., Kargel, J. S., Koppes, M.,
Kraaijenbrink, P. D. A., Kulkarni, A. V., Mayewski, P. A., \ldots{}
Baillie, J. E. M. (2020). Importance and vulnerability of the world's
water towers. \emph{Nature}, \emph{577}(7790), 364--369.
\url{https://doi.org/10.1038/s41586-019-1822-y}

\bibitem[\citeproctext]{ref-Katsushima2023}
Katsushima, T., Kato, A., Aiura, H., Nanko, K., Suzuki, S., Takeuchi,
Y., \& Murakami, S. (2023). Modelling of snow interception on a
{Japanese} cedar canopy based on weighing tree experiment in a warm
winter region. \emph{Hydrological Processes}, \emph{37}(6), 1--16.
\url{https://doi.org/10.1002/hyp.14922}

\bibitem[\citeproctext]{ref-Kesselring2024}
Kesselring, J., Morsdorf, F., Kükenbrink, D., Gastellu-Etchegorry,
J.-P., \& Damm, A. (2024). Diversity of {3D APAR} and {LAI} dynamics in
broadleaf and coniferous forests: {Implications} for the interpretation
of remote sensing-based products. \emph{Remote Sensing of Environment},
\emph{306}, 114116. \url{https://doi.org/10.1016/j.rse.2024.114116}

\bibitem[\citeproctext]{ref-Kim2017}
Kim, E., Gatebe, C., Hall, D., Newlin, J., Misakonis, A., Elder, K.,
Marshall, H. P., Hiemstra, C., Brucker, L., De Marco, E., Crawford, C.,
Kang, D. H., \& Entin, J. (2017). {NASA}'s snowex campaign: {Observing}
seasonal snow in a forested environment. \emph{2017 {IEEE} International
Geoscience and Remote Sensing Symposium ({IGARSS})}, 1388--1390.
\url{https://doi.org/10.1109/IGARSS.2017.8127222}

\bibitem[\citeproctext]{ref-Kozak1995}
Kozak, A., \& Kozak, R. A. (1995). Notes on regression through the
origin. \emph{Forestry Chronicle}, \emph{71}(3), 326--330.
\url{https://doi.org/10.5558/tfc71326-3}

\bibitem[\citeproctext]{ref-Krinner2018}
Krinner, G., Derksen, C., Essery, R., Flanner, M., Hagemann, S., Clark,
M. P., Hall, A., Rott, H., Brutel-Vuilmet, C., Kim, H., Ménard, C. B.,
Mudryk, L., Thackeray, C., Wang, L., Arduini, G., Balsamo, G., Bartlett,
P., Boike, J., Boone, A., \ldots{} Zhu, D. (2018). {ESM-SnowMIP}:
{Assessing} snow models and quantifying snow-related climate feedbacks.
\emph{Geoscientific Model Development}, \emph{11}(12), 5027--5049.
\url{https://doi.org/10.5194/gmd-11-5027-2018}

\bibitem[\citeproctext]{ref-Lopez-Moreno2014}
López-Moreno, J. I., Zabalza, J., Vicente-Serrano, S. M., Revuelto, J.,
Gilaberte, M., Azorin-Molina, C., Morán-Tejeda, E., García-Ruiz, J. M.,
\& Tague, C. (2014). Impact of climate and land use change on water
availability and reservoir management: {Scenarios} in the {Upper
Arag{ó}n River}, {Spanish Pyrenees}. \emph{Science of The Total
Environment}, \emph{493}, 1222--1231.
\url{https://doi.org/10.1016/j.scitotenv.2013.09.031}

\bibitem[\citeproctext]{ref-Lumbrazo2022}
Lumbrazo, C., Bennett, A., Currier, W. R., Nijssen, B., \& Lundquist, J.
(2022). Evaluating multiple canopy-snow unloading parameterizations in
{SUMMA} with time-lapse photography characterized by citizen scientists.
\emph{Water Resources Research}, \emph{58}(6), 1--22.
\url{https://doi.org/10.1029/2021WR030852}

\bibitem[\citeproctext]{ref-Lundquist2021}
Lundquist, J. D., Dickerson-Lange, S., Gutmann, E., Jonas, T., Lumbrazo,
C., \& Reynolds, D. (2021). Snow interception modelling: {Isolated}
observations have led to many land surface models lacking appropriate
temperature sensitivities. \emph{Hydrological Processes}, \emph{35}(7),
1--20. \url{https://doi.org/10.1002/hyp.14274}

\bibitem[\citeproctext]{ref-MacDonald2010}
MacDonald, J. P. (2010). \emph{Unloading of intercepted snow in conifer
forests} (p. 93) {[}M.\{\{Sc\}\}. Thesis{]}. Department of Geography,
University of Saskatchewan.

\bibitem[\citeproctext]{ref-Parviainen2000}
Parviainen, J., \& Pomeroy, J. W. (2000). Multiple-scale modelling of
forest snow sublimation: {Initial} findings. \emph{Hydrological
Processes}, \emph{14}(15), 2669--2681.
\url{https://doi.org/10.1002/1099-1085(20001030)14:15\%3C2669::AID-HYP85\%3E3.0.CO;2-Q}

\bibitem[\citeproctext]{ref-Pomeroy2022}
Pomeroy, J. W., Brown, T., Fang, X., Shook, K. R., Pradhananga, D.,
Armstrong, R., Harder, P., Marsh, C., Costa, D., Krogh, S. A.,
Aubry-Wake, C., Annand, H., Lawford, P., He, Z., Kompanizare, M., \&
Moreno, J. I. L. (2022). The cold regions hydrological modelling
platform for hydrological diagnosis and prediction based on process
understanding. \emph{Journal of Hydrology}, \emph{615}(128711), 1--25.
\url{https://doi.org/10.1016/j.jhydrol.2022.128711}

\bibitem[\citeproctext]{ref-Pomeroy1995}
Pomeroy, J. W., \& Gray, D. M. (1995). \emph{Snowcover {Accumulation},
{Relocation} and {Management}} (NHRI Science Report No. 7, p. 144).
National Hydrology Research Institute, Environment Canada, Saskatoon,
Canada.

\bibitem[\citeproctext]{ref-Pomeroy2002}
Pomeroy, J. W., Gray, D. M., Hedstrom, N. R., \& Janowicz, J. R. (2002).
Prediction of seasonal snow accumulation in cold climate forests.
\emph{Hydrological Processes}, \emph{16}(18), 3543--3558.
\url{https://doi.org/10.1002/hyp.1228}

\bibitem[\citeproctext]{ref-Pomeroy1998c}
Pomeroy, J. W., Gray, D. M., Shook, K. R., Toth, B., Essery, R. L. H.
H., Pietroniro, A., \& Hedstrom, N. (1998a). An evaluation of snow
accumulation and ablation processes for land surface modelling.
\emph{Hydrological Processes}, \emph{12}(15), 2339--2367.
\url{https://doi.org/10.1002/(SICI)1099-1085(199812)12:15\%3C2339::AID-HYP800\%3E3.0.CO;2-L}

\bibitem[\citeproctext]{ref-Pomeroy2009}
Pomeroy, J. W., Marks, D., Link, T., Ellis, C. R., Hardy, J., Rowlands,
A., \& Granger, R. (2009). The impact of coniferous forest temperature
on incoming longwave radiation to melting snow. \emph{Hydrological
Processes}, \emph{23}, 2513--2525.
\url{https://doi.org/10.1002/hyp.7325}

\bibitem[\citeproctext]{ref-Pomeroy1998b}
Pomeroy, J. W., Parviainen, J., Hedstrom, N., \& Gray, D. M. (1998b).
Coupled modelling of forest snow interception and sublimation.
\emph{Hydrological Processes}, \emph{12}(15), 2317--2337.
\url{https://doi.org/10.1002/(SICI)1099-1085(199812)12:15\%3C2317::AID-HYP799\%3E3.0.CO;2-X}

\bibitem[\citeproctext]{ref-Pomeroy1993a}
Pomeroy, J. W., \& Schmidt, R. A. (1993). The use of fractal geometry in
modelling intercepted snow accumulation and sublimation. \emph{Eastern
Snow Conference}, \emph{50}, 231--239.

\bibitem[\citeproctext]{ref-Roesch2001}
Roesch, A., Wild, M., Gilgen, H., \& Ohmura, A. (2001). A new snow cover
fraction parameterization for the {ECHAM4 GCM}. \emph{Climate Dynamics},
\emph{17}(12), 933--946. \url{https://doi.org/10.1007/s003820100153}

\bibitem[\citeproctext]{ref-Rutter2009}
Rutter, N., Essery, R., Pomeroy, J. W., Altimir, N., Andreadis, K. M.,
Baker, I., Barr, A., Bartlett, P., Boone, A., Deng, H., Douville, H.,
Dutra, E., Elder, K., Ellis, C. R., Feng, X., Gelfan, A., Goodbody, A.,
Gusev, Y., Gustafsson, D., \ldots{} Yamazaki, T. (2009). Evaluation of
forest snow processes models ({SnowMIP2}). \emph{Journal of Geophysical
Research: Atmospheres}, \emph{114}(D6), 10--18.
\url{https://doi.org/10.1029/2008JD011063}

\bibitem[\citeproctext]{ref-Safa2021}
Safa, H., Krogh, S. A., Greenberg, J., Kostadinov, T. S., \& Harpold, A.
A. (2021). Unraveling the controls on snow disappearance in montane
conifer forests using multi-site lidar. \emph{Water Resources Research},
\emph{57}(12), 1--20. \url{https://doi.org/10.1029/2020WR027522}

\bibitem[\citeproctext]{ref-Sanmiguel-Vallelado2020}
Sanmiguel-Vallelado, A., López-Moreno, J. I., Morán-Tejeda, E.,
Alonso-González, E., Navarro-Serrano, F. M., Rico, I., \& Camarero, J.
J. (2020). Variable effects of forest canopies on snow processes in a
valley of the central {Spanish Pyrenees}. \emph{Hydrological Processes},
\emph{34}(10), 2247--2262. \url{https://doi.org/10.1002/hyp.13721}

\bibitem[\citeproctext]{ref-Sanmiguel-Vallelado2022}
Sanmiguel-Vallelado, A., McPhee, J., Esmeralda Ojeda Carreño, P.,
Morán-Tejeda, E., Julio Camarero, J., \& López-Moreno, J. I. (2022).
Sensitivity of forest--snow interactions to climate forcing: {Local}
variability in a {Pyrenean} valley. \emph{Journal of Hydrology},
\emph{605}. \url{https://doi.org/10.1016/j.jhydrol.2021.127311}

\bibitem[\citeproctext]{ref-Schmidt1991}
Schmidt, R. A., \& Gluns, D. R. (1991). Snowfall interception on
branches of three conifer species. \emph{Canadian Journal of Forest
Research}, \emph{21}(8), 1262--1269.
\url{https://doi.org/10.1139/x91-176}

\bibitem[\citeproctext]{ref-Schmidt1990}
Schmidt, R. A., \& Pomeroy, J. W. (1990). Bending of a conifer branch at
subfreezing temperatures: Implications for snow interception.
\emph{Canadian Journal of Forest Research}, \emph{20}(8), 1251--1253.
\url{https://doi.org/10.1139/x90-165}

\bibitem[\citeproctext]{ref-Sicart2006}
Sicart, J. E., Pomeroy, J. W., Essery, R. L. H., \& Bewley, D. (2006).
Incoming longwave radiation to melting snow: Observations, sensitivity
and estimation in {Northern} environments. \emph{Hydrological
Processes}, \emph{20}(17), 3697--3708.

\bibitem[\citeproctext]{ref-Staines2023}
Staines, J., \& Pomeroy, J. W. (2023). Influence of forest canopy
structure and wind flow on patterns of sub-canopy snow accumulation in
montane needleleaf forests. \emph{Hydrological Processes},
\emph{37}(10), 1--19. \url{https://doi.org/10.1002/hyp.15005}

\bibitem[\citeproctext]{ref-Storck2002}
Storck, P., Lettenmaier, D. P., \& Bolton, S. M. (2002). Measurement of
snow interception and canopy effects on snow accumulation and melt in a
mountainous maritime climate, {Oregon}, {United States}. \emph{Water
Resources Research}, \emph{38}(11), 1--16.
\url{https://doi.org/10.1029/2002wr001281}

\bibitem[\citeproctext]{ref-Szczypta2015}
Szczypta, C., Gascoin, S., Houet, T., Hagolle, O., Dejoux, J.-F.,
Vigneau, C., \& Fanise, P. (2015). Impact of climate and land cover
changes on snow cover in a small {Pyrenean} catchment. \emph{Journal of
Hydrology}, \emph{521}, 84--99.
\url{https://doi.org/10.1016/j.jhydrol.2014.11.060}

\bibitem[\citeproctext]{ref-Troendle1983}
Troendle, C. A. (1983). The potential for water yield augmentation from
forest management in the {Rocky Mountain} region. \emph{Journal of the
American Water Resources Association}, \emph{19}(3), 359--373.
\url{https://doi.org/10.1111/j.1752-1688.1983.tb04593.x}

\bibitem[\citeproctext]{ref-Valante1997}
Valante, F., David, J. S., \& Gash, J. H. C. (1997). Modelling
interception loss for two sparse eucalypt and pine forests in central
{Portugal} using reformulated {Rutter} and {Gash} analytical models.
\emph{Journal of Hydrology}, \emph{190}(1-2), 141--162.
\url{https://doi.org/10.1016/S0022-1694(96)03066-1}

\bibitem[\citeproctext]{ref-Verseghy2017}
Verseghy, D. L. (2017). \emph{Class -- the {Canadian Land Surface
Scheme} (version 3.6.1) technical documentation.} (January; p. 174).
{Environment and Climate Change Canada Internal Rep.}

\bibitem[\citeproctext]{ref-Viviroli2020}
Viviroli, D., Kummu, M., Meybeck, M., Kallio, M., \& Wada, Y. (2020).
Increasing dependence of lowland populations on mountain water
resources. \emph{Nature Sustainability}, \emph{3}(11), 917--928.
\url{https://doi.org/10.1038/s41893-020-0559-9}

\bibitem[\citeproctext]{ref-Weiskittel2009}
Weiskittel, A. R., Kershaw, J. A., Hofmeyer, P. V., \& Seymour, R. S.
(2009). Species differences in total and vertical distribution of
branch- and tree-level leaf area for the five primary conifer species in
{Maine}, {USA}. \emph{Forest Ecology and Management}, \emph{258}(7),
1695--1703. \url{https://doi.org/10.1016/j.foreco.2009.07.035}

\bibitem[\citeproctext]{ref-Wheater2022}
Wheater, H. S., Pomeroy, J. W., Pietroniro, A., Davison, B., Elshamy,
M., Yassin, F., Rokaya, P., Fayad, A., Tesemma, Z., Princz, D., Loukili,
Y., DeBeer, C. M., Ireson, A. M., Razavi, S., Lindenschmidt, K.-E.,
Elshorbagy, A., MacDonald, M., Abdelhamed, M., Haghnegahdar, A., \&
Bahrami, A. (2022). Advances in modelling large river basins in cold
regions with {Mod{é}lisation Environmentale Communautaire}---{Surface}
and {Hydrology} ({MESH}), the {Canadian} hydrological land surface
scheme. \emph{Hydrological Processes}, \emph{36}(4), 1--24.
\url{https://doi.org/10.1002/hyp.14557}

\bibitem[\citeproctext]{ref-Zhong2022}
Zhong, F., Jiang, S., van Dijk, A. I. J. M., Ren, L., Schellekens, J.,
\& Miralles, D. G. (2022). Revisiting large-scale interception patterns
constrained by a synthesis of global experimental data. \emph{Hydrology
and Earth System Sciences}, \emph{26}(21), 5647--5667.
\url{https://doi.org/10.5194/hess-26-5647-2022}

\end{CSLReferences}




\end{document}
